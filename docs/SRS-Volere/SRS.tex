% THIS DOCUMENT IS FOLLOWS THE VOLERE TEMPLATE BY Suzanne Robertson and James Robertson
% ONLY THE SECTION HEADINGS ARE PROVIDED
%
% Initial draft from https://github.com/Dieblich/volere
%
% Risks are removed because they are covered by the Hazard Analysis
\documentclass[12pt]{article}

\usepackage{booktabs}
\usepackage{tabularx}
\usepackage{hyperref}
\hypersetup{
    bookmarks=true,         % show bookmarks bar?
      colorlinks=true,      % false: boxed links; true: colored links
    linkcolor=red,          % color of internal links (change box color with linkbordercolor)
    citecolor=green,        % color of links to bibliography
    filecolor=magenta,      % color of file links
    urlcolor=cyan           % color of external links
}

\newcommand{\lips}{\textit{Insert your content here.}}

%% Comments

\usepackage{color}

\newif\ifcomments\commentstrue %displays comments
%\newif\ifcomments\commentsfalse %so that comments do not display

\ifcomments
\newcommand{\authornote}[3]{\textcolor{#1}{[#3 ---#2]}}
\newcommand{\todo}[1]{\textcolor{red}{[TODO: #1]}}
\else
\newcommand{\authornote}[3]{}
\newcommand{\todo}[1]{}
\fi

\newcommand{\wss}[1]{\authornote{blue}{SS}{#1}} 
\newcommand{\plt}[1]{\authornote{magenta}{TPLT}{#1}} %For explanation of the template
\newcommand{\an}[1]{\authornote{cyan}{Author}{#1}}

%% Common Parts

\newcommand{\progname}{Software Engineering} % PUT YOUR PROGRAM NAME HERE
\newcommand{\authname}{Team 2, SyntaxSentinals
\\ Lucas Chen
\\ Dennis Fong
\\ Mohammad Mohsin Khan
\\ Julian Cecchini
\\ Luigi Quattrociocchi} % AUTHOR NAMES                  

\usepackage{hyperref}
    \hypersetup{colorlinks=true, linkcolor=blue, citecolor=blue, filecolor=blue,
                urlcolor=blue, unicode=false}
    \urlstyle{same}
                                


\begin{document}

\title{Software Requirements Specification for \progname: Code Plagiarism Detector} 
\author{\authname}
\date{\today}
	
\maketitle

~\newpage

\pagenumbering{roman}

\tableofcontents

~\newpage

\section*{Revision History}

\begin{tabularx}{\textwidth}{p{3cm}p{2cm}X}
\toprule {\textbf{Date}} & {\textbf{Version}} & {\textbf{Notes}}\\
\midrule
Date 1 & 1.0 & Notes\\
Date 2 & 1.1 & Notes\\
\bottomrule
\end{tabularx}

~\\

~\newpage
\section{Purpose of the Project}
\subsection{User Business}
\lips
\subsection{Goals of the Project}
\lips
\section{Stakeholders}
\subsection{Client}
\lips
\subsection{Customer}
\lips
\subsection{Other Stakeholders}
\lips
\subsection{Hands-On Users of the Project}
\lips
\subsection{Personas}
\lips
\subsection{Priorities Assigned to Users}
\lips
\subsection{User Participation}
\lips
\subsection{Maintenance Users and Service Technicians}
\lips

\section{Mandated Constraints}
\subsection{Solution Constraints}
\lips
\subsection{Implementation Environment of the Current System}
\lips
\subsection{Partner or Collaborative Applications}
\lips
\subsection{Off-the-Shelf Software}
\lips
\subsection{Anticipated Workplace Environment}
\lips
\subsection{Schedule Constraints}
\lips
\subsection{Budget Constraints}
\lips
\subsection{Enterprise Constraints}
\lips

\section{Naming Conventions and Terminology}
\subsection{Glossary of All Terms, Including Acronyms, Used by Stakeholders
involved in the Project}
\lips

\section{Relevant Facts And Assumptions}
\subsection{Relevant Facts}
\lips
\subsection{Business Rules}
\lips
\subsection{Assumptions}
\lips

\section{The Scope of the Work}
\subsection{The Current Situation}
The current code plagiarism dtection tools such as MOSS rely on tokenization and syntax-level comparisons. Although effective
in detecting direct copies or slight variations, they struggle when faced with techniques such as adding redundant code which
allows the user to completely bypass detection while still plagiarizing the underlying logic and structure of the code.

Additionally, MOSS does not take into account for the complexity or intent behind the code, leading to issues such as false
positives for common programming patterns. This creates a gap for more advanced tools capable of understanding the semantic 
meaning of code to more accurately detect plagiarism.
\subsection{The Context of the Work}
Our project aims to address these gaps by incorporating Natural Language Processing (NLP) and machine learning techniques
which will be leveraged to improve the accuracy of detecting copied code. The context of the work is within academic institutions,
where the integrity of student work is aparmount and our tool will be used by professors to ensure a fair grading process
while also supporting students in understanding the ethical use of code.
\subsection{Work Partitioning}
\begin{itemize}
  \item \textbf{Research and Design}: Conduct research on current plagiarism detection systems and state-of-the-art NLP techniques applicable to code plagiarism.
  
  \item \textbf{Data Collection}: Gather a dataset of code snippets, including both plagiarized and original works, to train and test the model.
  
  \item \textbf{Model Development}: Develop the NLP-based model capable of understanding the semantic meaning of code. This may involve exploring techniques like abstract syntax trees (ASTs), vector embeddings, or other representations of code that retain semantic meaning.
  
  \item \textbf{System Integration}: Build the system to take code as input, run through the developed model, and output a similarity score with appropriate thresholds.
  
  \item \textbf{Testing and Validation}: Test the system with various code samples to validate its performance and accuracy compared to traditional systems like MOSS. This will also test wether our method produces any false positives.
  
  \item \textbf{Documentation and Deployment}: Document the system architecture, the model, and the results. Deploy the system for use within academic settings.
\end{itemize}
\subsection{Specifying a Business Use Case (BUC)}

\textbf{Business Use Case:} Automated Code Plagiarism Detection for Academic Institutions

\begin{itemize}
    \item \textbf{Actors:} Professors, Students, System Administrators
    \item \textbf{Trigger:} A professor or system administrator uploads multiple code submissions for plagiarism detection in a course assignment.
\end{itemize}

\textbf{Main Success Scenario}
\begin{enumerate}
    \item The system ingests the uploaded code submissions.
    \item The system processes each code snippet using the NLP model to generate semantic representations of the code.
    \item The system compares the representations to detect plagiarism, taking into account code similarity beyond syntax or token matching.
    \item The system outputs a similarity score for each comparison, with thresholds indicating whether plagiarism is suspected.
    \item The professor reviews the similarity scores and flags any suspicious cases for further investigation.
    \item The system generates a report summarizing the findings for the professor’s review.
\end{enumerate}

\textbf{Extensions}
\begin{itemize}
    \item If the system detects false positives (common programming patterns being flagged as plagiarism), the professor can override the result.
    \item If new sophisticated plagiarism techniques are detected, the system can update its learning algorithms to improve accuracy over time.
\end{itemize}

\section{Business Data Model and Data Dictionary}
\subsection{Business Data Model}
\lips
\subsection{Data Dictionary}
\lips

\section{The Scope of the Product}
\subsection{Product Boundary}
\lips
\subsection{Product Use Case Table}
\lips
\subsection{Individual Product Use Cases (PUC's)}
\lips

\section{Functional Requirements}
\subsection{Functional Requirements}
\lips

\section{Look and Feel Requirements}
\subsection{Appearance Requirements}
\lips
\subsection{Style Requirements}
\lips

\section{Usability and Humanity Requirements}
\subsection{Ease of Use Requirements}
\lips
\subsection{Personalization and Internationalization Requirements}
\lips
\subsection{Learning Requirements}
\lips
\subsection{Understandability and Politeness Requirements}
\lips
\subsection{Accessibility Requirements}
\lips

\section{Performance Requirements}
\subsection{Speed and Latency Requirements}
\lips
\subsection{Safety-Critical Requirements}
\lips
\subsection{Precision or Accuracy Requirements}
\lips
\subsection{Robustness or Fault-Tolerance Requirements}
\lips
\subsection{Capacity Requirements}
\lips
\subsection{Scalability or Extensibility Requirements}
\lips
\subsection{Longevity Requirements}
\lips

\section{Operational and Environmental Requirements}
\subsection{Expected Physical Environment}
\lips
\subsection{Wider Environment Requirements}
\lips
\subsection{Requirements for Interfacing with Adjacent Systems}
\lips
\subsection{Productization Requirements}
\lips
\subsection{Release Requirements}
\lips

\section{Maintainability and Support Requirements}
\subsection{Maintenance Requirements}
\lips
\subsection{Supportability Requirements}
\lips
\subsection{Adaptability Requirements}
\lips

\section{Security Requirements}
\subsection{Access Requirements}
\lips
\subsection{Integrity Requirements}
\lips
\subsection{Privacy Requirements}
\lips
\subsection{Audit Requirements}
\lips
\subsection{Immunity Requirements}
\lips

\section{Cultural Requirements}
\subsection{Cultural Requirements}
\lips

\section{Compliance Requirements}

In developing the enhanced plagiarism detection tool, it is imperative to address
various compliance requirements to ensure the tool operates legally, ethically,
and in alignment with industry standards. These requirements encompass legal
obligations related to data protection, intellectual property rights, and
adherence to educational policies, as well as compliance with established software
development and data security standards.

\subsection{Legal Requirements}

\begin{enumerate}
    \item \textbf{Data Protection and Privacy Laws}: The tool will process sensitive
    information, including students' code submissions, which may be considered personal
    data under Canadian privacy laws such as the \textit{Personal Information Protection 
    and Electronic Documents Act} (PIPEDA) at the federal level, and Ontario's \textit{Freedom of 
    Information and Protection of Privacy Act} (FIPPA) for public institutions. Compliance with these 
    laws requires:
    \begin{itemize}
        \item \textbf{Lawful Basis for Data Processing}: Ensuring that the collection and use of 
        personal information is authorized under PIPEDA or FIPPA, typically requiring consent from 
        students before processing their code or ensuring that processing is necessary for educational purposes.
        \item \textbf{Data Minimization and Purpose Limitation}: Collecting only the data
        necessary for plagiarism detection and using it solely for that purpose.
        \item \textbf{Transparency and Information Rights}: Informing students about how
        their data will be used, stored, and protected, and respecting their rights to
        access, correct, or withdraw their personal information.
        \item \textbf{Security Measures}: Implementing appropriate technical and
        organizational measures to safeguard personal data against unauthorized access,
        loss, or disclosure, as required under PIPEDA and FIPPA.
    \end{itemize}

    \item \textbf{Intellectual Property Rights}: Under the \textit{Copyright Act} of Canada, 
    students typically hold the intellectual property rights to their original code. 
    The tool must:
    \begin{itemize}
        \item \textbf{Respect Ownership}: Use students' code exclusively for plagiarism
        detection without unauthorized distribution or reproduction.
        \item \textbf{Establish Clear Terms}: Provide clear terms of service or agreements
        outlining how the code will be used, ensuring students are aware and consent to
        these terms.
        \item \textbf{Avoid Infringement}: Ensure that any storage or processing of code
        does not violate the \textit{Copyright Act} or institutional policies.
    \end{itemize}

    \item \textbf{Academic Integrity Policies}: The tool must align with the academic
    integrity and misconduct policies of Canadian educational institutions by:
    \begin{itemize}
        \item \textbf{Supporting Fair Evaluation}: Assisting educators in identifying
        potential plagiarism accurately without bias.
        \item \textbf{Due Process}: Ensuring that students have the opportunity to respond
        to plagiarism accusations, with results from the tool serving as part of a broader
        investigation rather than definitive proof.
        \item \textbf{Confidentiality}: Maintaining the confidentiality of students' work
        and any findings related to plagiarism investigations.
    \end{itemize}
\end{enumerate}

\subsection{Standards Compliance Requirements}

\begin{enumerate}
    \item \textbf{Software Development Standards}: Adherence to recognized software
    development practices and standards is essential for ensuring quality and reliability.
    \begin{itemize}
        \item \textbf{ISO/IEC 25010 Compliance}: Aligning with the ISO/IEC 25010 standard
        for software product quality, focusing on functionality, reliability, usability,
        efficiency, maintainability, and portability.
        \item \textbf{Documentation and Testing}: Maintaining thorough documentation and
        conducting rigorous testing to validate the tool's performance and reliability.
    \end{itemize}

    \item \textbf{Data Security Standards}: Protecting sensitive data requires compliance
    with established security standards.
    \begin{itemize}
        \item \textbf{OWASP Guidelines}: Implementing security measures in line with the
        Open Web Application Security Project (OWASP) guidelines to prevent common
        vulnerabilities such as injection attacks, data breaches, and unauthorized access.
        \item \textbf{ISO/IEC 27001 Certification}: Considering certification under the
        ISO/IEC 27001 standard for information security management to demonstrate a
        commitment to data security best practices.
    \end{itemize}

    \item \textbf{Accessibility Standards}: The tool should be accessible to all users,
    including those with disabilities.
    \begin{itemize}
        \item \textbf{AODA Compliance}: Designing the user interface in accordance
        with the \textit{Accessibility for Ontarians with Disabilities Act} (AODA) and the 
        \textit{Integrated Accessibility Standards Regulation} (IASR) to ensure it is
        perceivable, operable, understandable, and robust for all users.
        \item \textbf{WCAG 2.1 Compliance}: Ensuring that the tool meets the Web Content 
        Accessibility Guidelines (WCAG) 2.1 Level AA standards, as required under AODA.
    \end{itemize}

    \item \textbf{Ethical AI and Machine Learning Standards}: As the tool leverages AI
    technologies, it must adhere to ethical standards in AI development.
    \begin{itemize}
        \item \textbf{Transparency and Explainability}: Ensuring that the AI models used
        are transparent in their operation and that their decision-making processes can be
        explained to users.
        \item \textbf{Fairness and Non-Discrimination}: Preventing biases in the AI models
        that could unfairly target or disadvantage any group of students.
        \item \textbf{Canadian AI Ethical Guidelines}: Following principles outlined in the 
        \textit{Directive on Automated Decision-Making} by the Government of Canada and 
        guidelines from organizations such as the \textit{Canadian Institute for Advanced Research} (CIFAR) 
        for promoting ethical considerations in AI design and deployment.
    \end{itemize}

    \item \textbf{Data Handling and Retention Policies}: Establishing clear policies for
    how data is managed throughout its lifecycle.
    \begin{itemize}
        \item \textbf{Retention Limits}: Defining how long code submissions and related
        data will be stored, in compliance with PIPEDA, FIPPA, and institutional policies.
        \item \textbf{Secure Disposal}: Implementing procedures for the secure deletion or
        anonymization of data that is no longer needed.
        \item \textbf{Audit and Compliance}: Regularly auditing data handling practices to
        ensure ongoing compliance with all relevant laws and standards.
    \end{itemize}
\end{enumerate}

By meticulously addressing these legal and standards compliance requirements, the
project not only safeguards the rights and interests of all stakeholders but also
enhances the credibility and trustworthiness of the plagiarism detection tool. Ensuring
compliance is fundamental to the tool's success and its acceptance by educational
institutions, educators, and students alike.


\section{Open Issues}
\lips

\section{Off-the-Shelf Solutions}
\subsection{Ready-Made Products}
\lips
\subsection{Reusable Components}
\lips
\subsection{Products That Can Be Copied}
\lips

\section{New Problems}
\subsection{Effects on the Current Environment}
\lips
\subsection{Effects on the Installed Systems}
\lips
\subsection{Potential User Problems}
\lips
\subsection{Limitations in the Anticipated Implementation Environment That May
Inhibit the New Product}
\lips
\subsection{Follow-Up Problems}
\lips

\section{Tasks}
\subsection{Project Planning}
\lips
\subsection{Planning of the Development Phases}
\lips

\section{Migration to the New Product}
\subsection{Requirements for Migration to the New Product}
\lips
\subsection{Data That Has to be Modified or Translated for the New System}
\lips

\section{Costs}
\lips
\section{User Documentation and Training}
\subsection{User Documentation Requirements}
\lips
\subsection{Training Requirements}
\lips

\section{Waiting Room}
\lips

\section{Ideas for Solution}
\lips

\newpage{}
\section*{Appendix --- Reflection}

The information in this section will be used to evaluate the team members on the
graduate attribute of Lifelong Learning.  Please answer the following questions:

\begin{enumerate}
  \item What knowledge and skills will the team collectively need to acquire to
  successfully complete this capstone project?  Examples of possible knowledge
  to acquire include domain specific knowledge from the domain of your
  application, or software engineering knowledge, mechatronics knowledge or
  computer science knowledge.  Skills may be related to technology, or writing,
  or presentation, or team management, etc.  You should look to identify at
  least one item for each team member.
  \item For each of the knowledge areas and skills identified in the previous
  question, what are at least two approaches to acquiring the knowledge or
  mastering the skill?  Of the identified approaches, which will each team
  member pursue, and why did they make this choice?
\end{enumerate}

\end{document}
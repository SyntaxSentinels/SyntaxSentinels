\documentclass[12pt, titlepage]{article}

\usepackage{booktabs}
\usepackage{tabularx}
\usepackage{hyperref}
\hypersetup{
    colorlinks,
    citecolor=blue,
    filecolor=black,
    linkcolor=red,
    urlcolor=blue
}
\usepackage[round]{natbib}

%% Comments

\usepackage{color}

\newif\ifcomments\commentstrue %displays comments
%\newif\ifcomments\commentsfalse %so that comments do not display

\ifcomments
\newcommand{\authornote}[3]{\textcolor{#1}{[#3 ---#2]}}
\newcommand{\todo}[1]{\textcolor{red}{[TODO: #1]}}
\else
\newcommand{\authornote}[3]{}
\newcommand{\todo}[1]{}
\fi

\newcommand{\wss}[1]{\authornote{blue}{SS}{#1}} 
\newcommand{\plt}[1]{\authornote{magenta}{TPLT}{#1}} %For explanation of the template
\newcommand{\an}[1]{\authornote{cyan}{Author}{#1}}

%% Common Parts

\newcommand{\progname}{Software Engineering} % PUT YOUR PROGRAM NAME HERE
\newcommand{\authname}{Team 2, SyntaxSentinals
\\ Lucas Chen
\\ Dennis Fong
\\ Mohammad Mohsin Khan
\\ Julian Cecchini
\\ Luigi Quattrociocchi} % AUTHOR NAMES                  

\usepackage{hyperref}
    \hypersetup{colorlinks=true, linkcolor=blue, citecolor=blue, filecolor=blue,
                urlcolor=blue, unicode=false}
    \urlstyle{same}
                                


\begin{document}

\title{System Verification and Validation Plan for \progname{}} 
\author{\authname}
\date{\today}
	
\maketitle

\pagenumbering{roman}

\section*{Revision History}

\begin{tabularx}{\textwidth}{p{3cm}p{2cm}X}
\toprule {\bf Date} & {\bf Version} & {\bf Notes}\\
\midrule
November 1 & 1.0 & Initial documentation\\
% Date 2 & 1.1 & Notes\\
\bottomrule
\end{tabularx}

~\\
\wss{The intention of the VnV plan is to increase confidence in the software.
However, this does not mean listing every verification and validation technique
that has ever been devised.  The VnV plan should also be a \textbf{feasible}
plan. Execution of the plan should be possible with the time and team available.
If the full plan cannot be completed during the time available, it can either be
modified to ``fake it'', or a better solution is to add a section describing
what work has been completed and what work is still planned for the future.}

\wss{The VnV plan is typically started after the requirements stage, but before
the design stage.  This means that the sections related to unit testing cannot
initially be completed.  The sections will be filled in after the design stage
is complete.  the final version of the VnV plan should have all sections filled
in.}

\newpage

\tableofcontents

\listoftables
\wss{Remove this section if it isn't needed}

\listoffigures
\wss{Remove this section if it isn't needed}

\newpage

\section{Symbols, Abbreviations, and Acronyms}

\renewcommand{\arraystretch}{1.2}
\begin{tabular}{l l} 
  \toprule		
  \textbf{symbol} & \textbf{description}\\
  \midrule 
  T & Test\\
  \bottomrule
\end{tabular}\\

\wss{symbols, abbreviations, or acronyms --- you can simply reference the SRS
  \citep{SRS} tables, if appropriate}

\wss{Remove this section if it isn't needed}

\newpage

\pagenumbering{arabic}

% This document ... \wss{provide an introductory blurb and roadmap of the
%   Verification and Validation plan}
This document outlines the strategies and processes used to ensure that the plagiarism detection system  developed by the SyntaxSentinals team meets all functional 
and non-functional requirements. The primary goal of this plan is to build confidence in the correctness, usability, and performance of the system. It also focuses 
on identifying and mitigating potential risks, ensuring that the final product aligns with academic and competition standards.
This document is organized as follows. The general information section provides an overview of the objectives, challenges, and relevant project documents 
used throughout the V\&V process. The plan section describes the roles and responsibilities of the team members and the tools used for automated testing and verification. 
The system tests section lists the tests performed for both functional and non-functional requirements, with traceability to the SRS. 
The unit test description section details the unit testing scope, the modules tested, and the strategies for covering edge cases. Finally, the appendix contains symbolic parameters, 
survey questions (if applicable), and any other relevant information to support the V\&V process. This plan will evolve as the project progresses, with updates following the completion 
of the detailed design and implementation phases.

\section{General Information}

\subsection{Summary}

% \wss{Say what software is being tested.  Give its name and a brief overview of
%   its general functions.}
  The software being tested is a plagiarism detection system designed to identify similarities in Python code submissions, called SyntaxSentinels. 
  This system utilizes natural language processing (NLP) techniques to analyze code semantics, preventing common circumvention methods 
  like adding benign lines or altering variable names. Its core function is to allow users to input code snippets and receive a plagiarism 
  report containing similarity scores, which, when compared to a threshold, indicate the likelihood of plagiarism. This tool is primarily 
  intended for use in academic and competitive environments to promote fairness and integrity in code submissions.

\subsection{Objectives}

% \wss{State what is intended to be accomplished.  The objective will be around
%   the qualities that are most important for your project.  You might have
%   something like: ``build confidence in the software correctness,''
%   ``demonstrate adequate usability.'' etc.  You won't list all of the qualities,
%   just those that are most important.}
The primary objectives of this V\&V plan are to:
\begin{itemize}
  \item Build confidence in the programs correctness by ensuring alignment with the SRS requirements.
  \item Show that we have met the documented safety and security requirements (SR-SAF1- SR-SAF5) in the Hazard Analysis document.
  \item Demonstrate adequate usability in the program by conducting functional and non-functional tests mentioned in this document.
\end{itemize}

Out of Scope:
\begin{itemize}
  \item Validation of any external libraries will be assumed to be handled by their maintainers.
\end{itemize}

% \wss{You should also list the objectives that are out of scope.  You don't have 
% the resources to do everything, so what will you be leaving out.  For instance, 
% if you are not going to verify the quality of usability, state this.  It is also 
% worthwhile to justify why the objectives are left out.}

% \wss{The objectives are important because they highlight that you are aware of 
% limitations in your resources for verification and validation.  You can't do everything, 
% so what are you going to prioritize?  As an example, if your system depends on an 
% external library, you can explicitly state that you will assume that external library 
% has already been verified by its implementation team.}

\subsection{Challenge Level and Extras}

% \wss{State the challenge level (advanced, general, basic) for your project.
% Your challenge level should exactly match what is included in your problem
% statement.  This should be the challenge level agreed on between you and the
% course instructor.  You can use a pull request to update your challenge level
% (in TeamComposition.csv or Repos.csv) if your plan changes as a result of the
% VnV planning exercise.}

% \wss{Summarize the extras (if any) that were tackled by this project.  Extras
% can include usability testing, code walkthroughs, user documentation, formal
% proof, GenderMag personas, Design Thinking, etc.  Extras should have already
% been approved by the course instructor as included in your problem statement.
% You can use a pull request to update your extras (in TeamComposition.csv or
% Repos.csv) if your plan changes as a result of the VnV planning exercise.}

This project has been classified as having a General difficulty level, as agreed with the course instructor. 
Planned extras include:
\begin{itemize}
  \item A user manual for instructors and administrators.
  \item Benchmarking of the tool’s effectiveness compared to MOSS.
\end{itemize}

\subsection{Relevant Documentation}

% \wss{Reference relevant documentation.  This will definitely include your SRS
%   and your other project documents (design documents, like MG, MIS, etc).  You
%   can include these even before they are written, since by the time the project
%   is done, they will be written.  You can create BibTeX entries for your
%   documents and within those entries include a hyperlink to the documents.}

% \citet{SRS}

% \wss{Don't just list the other documents.  You should explain why they are relevant and 
% how they relate to your VnV efforts.}

The following documents are critical to the development and V\&V efforts for this project.

\begin{itemize}
    \item \textbf{Software Requirements Specification (SRS)}: Defines the project’s requirements, guiding both verification and validation. ~\citep{SRS}.
    \item \textbf{User Guide}: Provides operational instructions, relevant for usability testing. ~\citep{UserGuide}.
    \item \textbf{Module Guide (MG)}: Outlines the system’s architecture, essential for design verification. ~\citep{MG}.
    \item \textbf{Module Interface Specification (MIS)}: Details the internal modules and interfaces, critical for unit testing. ~\citep{MIS}.
    \item \textbf{Hazard Analysis}: Identifies potential risks, guiding validation efforts for safety and security. ~\citep{HazardAnalysis}.
\end{itemize}

\section{Plan}

\wss{Introduce this section.  You can provide a roadmap of the sections to
  come.}

\subsection{Verification and Validation Team}

\wss{Your teammates.  Maybe your supervisor.
  You should do more than list names.  You should say what each person's role is
  for the project's verification.  A table is a good way to summarize this information.}

\subsection{SRS Verification Plan}

\wss{List any approaches you intend to use for SRS verification.  This may
  include ad hoc feedback from reviewers, like your classmates (like your
  primary reviewer), or you may plan for something more rigorous/systematic.}

\wss{If you have a supervisor for the project, you shouldn't just say they will
read over the SRS.  You should explain your structured approach to the review.
Will you have a meeting?  What will you present?  What questions will you ask?
Will you give them instructions for a task-based inspection?  Will you use your
issue tracker?}

\wss{Maybe create an SRS checklist?}

\subsection{Design Verification Plan}

\wss{Plans for design verification}

\wss{The review will include reviews by your classmates}

\wss{Create a checklists?}

\subsection{Verification and Validation Plan Verification Plan}

\wss{The verification and validation plan is an artifact that should also be
verified.  Techniques for this include review and mutation testing.}

\wss{The review will include reviews by your classmates}

\wss{Create a checklists?}

\subsection{Implementation Verification Plan}

\wss{You should at least point to the tests listed in this document and the unit
  testing plan.}

\wss{In this section you would also give any details of any plans for static
  verification of the implementation.  Potential techniques include code
  walkthroughs, code inspection, static analyzers, etc.}

\wss{The final class presentation in CAS 741 could be used as a code
walkthrough.  There is also a possibility of using the final presentation (in
CAS741) for a partial usability survey.}

\subsection{Automated Testing and Verification Tools}

\wss{What tools are you using for automated testing.  Likely a unit testing
  framework and maybe a profiling tool, like ValGrind.  Other possible tools
  include a static analyzer, make, continuous integration tools, test coverage
  tools, etc.  Explain your plans for summarizing code coverage metrics.
  Linters are another important class of tools.  For the programming language
  you select, you should look at the available linters.  There may also be tools
  that verify that coding standards have been respected, like flake9 for
  Python.}

\wss{If you have already done this in the development plan, you can point to
that document.}

\wss{The details of this section will likely evolve as you get closer to the
  implementation.}

\subsection{Software Validation Plan}

\wss{If there is any external data that can be used for validation, you should
  point to it here.  If there are no plans for validation, you should state that
  here.}

\wss{You might want to use review sessions with the stakeholder to check that
the requirements document captures the right requirements.  Maybe task based
inspection?}

\wss{For those capstone teams with an external supervisor, the Rev 0 demo should 
be used as an opportunity to validate the requirements.  You should plan on 
demonstrating your project to your supervisor shortly after the scheduled Rev 0 demo.  
The feedback from your supervisor will be very useful for improving your project.}

\wss{For teams without an external supervisor, user testing can serve the same purpose 
as a Rev 0 demo for the supervisor.}

\wss{This section might reference back to the SRS verification section.}

\section{System Tests}

Below are tests for functional and non-functional requirements stated in the 
relevant SRS  \citep{SRS}. 

\subsection{Tests for Functional Requirements}

Each test section addresses a particular initial condition and input set. 
Therefore, each test has a particular set of outputs that the system should 
produce for it. A functional requirement of section 9 in the SRS \citep{SRS}
is only associated with a test if the output set of the test corresponds to 
the fit criterion of that functional requirement, and the fit criterion should 
be an observable component of a system state, such as a display of a return 
item or notification presented. Through this association, every functional 
requirement is provided a test that is capable of verifying it, thereby ensuring 
the test areas fully cover the functional requirements.

\subsubsection{Test Area 1}

Applies to all FRs involving input of the system necessary for the plagiarism 
analysis. This currently covers FR-1 in subsection 9.1 of the SRS \citep{SRS}.

\paragraph{Input For Analysis Test}

\begin{enumerate}

\item{test-FR1-1\\}

Control: Manual.
					
Initial State: system is active, spot for giving system code snippet is open
with zero or more existing snippets currently added.
					
Input: code snippet file(s).
					
Output: System signifies input was received and continues activity without 
error, awaiting further action directives (such as giving more files or 
initiating analysis).

Test Case Derivation: The system will look to continually receive code 
snippets from user as part of a necessary step to initiate plagiarism analysis.
					
How test will be performed: system will be opened in the appropriate spot for 
receiving code and receive files in different amounts, one at a time, 
and will be inspected for failure in this process (types: dynamic, functional).
					

\end{enumerate}

\subsubsection{Test Area 2}

Applies to all FRs involving immediate output of plagiarism analysis. This 
currently covers FR-2, FR-4, and FR-5 in subsection 9.1 of the SRS \citep{SRS}.

\paragraph{Typical Detection Result Test}

\begin{enumerate}

\item{test-FR2,4,5-1\\}

Control: Manual.
					
Initial State: system is active, has received 2 or more code snippets, and 
spot for initiating analysis is open.
					
Input: code snippet files and command to initiate analysis (button, enter, 
etc.).
					
Output: System presents list of similarity scores for all code snippet pairings along 
with corresponding threshold scores in a viewable fashion to user and any code snippet
pairing that has exceeded its threshold has been flagged. System remains active, 
awaiting for action directives on what to do.

Test Case Derivation: The system should possess an algorithm that analyzes code 
snippets and produces similarity scores and threshold scores in return. These 
scores should be produced between every pairing of code snippets as the 
plagiarism is a relative assessment between one code piece and another. The 
system must able to pass this algorithm received code snippets and use its
results to flag any pairing that exceeded its threshold before returning from
its analysis state to the user for further interaction.

How test will be performed: system will be opened in spot to initiate analysis 
with 2 or more code snippets already inserted into it, and command for 
starting analysis will be passed from which point errors will be inspected for 
up until similarity scores and thresholds have been presented in viewable form, 
from which point code flaggings should also be available to ascertain. (type: 
dynamic, functional).
					

\end{enumerate}

\subsubsection{Test Area 3}

Applies to all FRs involving guide documentation generation for user. This 
currently covers FR-3 in subsection 9.1 of the SRS \citep{SRS}.

\paragraph{Guide Documentation Generation Test}

\begin{enumerate}

\item{test-FR3-1\\}

Control: Manual.
					
Initial State: System is active and spot for generating guide documentation is 
open.
					
Input: command for documentation presentation (button, enter, etc.).
					
Output: System presents documentation in viewable form to user and remains 
active awaiting for further action directives.

Test Case Derivation: system possesses guide documentation which it should be 
capable of transferring to user to provide guidance on use cases of system 
when prompted.

How test will be performed: system will be opened in spot for generating 
guide documentation and command for guide generating documentation will be 
passed, all the while inspecting for errors or missing documentation (type: 
dynamic, functional).				

\end{enumerate}

\subsubsection{Test Area 4}

Applies to all FRs involving analysis report documentation generation for 
user. This currently covers FR-6 of subsection 9.1 of the SRS \citep{SRS}.

\paragraph{Report Documentation Generation Test}

\begin{enumerate}

\item{test-FR6-1\\}

Control: Manual.
					
Initial State: System is active, most recent plagiarism analysis has been 
completed since system activation and its results are available, and spot 
for report generation is open.
					
Input: command for report documentation generation (button, enter, etc.).
					
Output: System provides report documentation in viewable form to user and 
remains active awaiting for further action directives.

Test Case Derivation: The system should possess the ability to aggregate
the results of the plagiarism analysis which has most recently occurred 
and insert them into a template that can summarize all findings to a user,
which will become the report document given.

How test will be performed: Proceeding a successful analysis run from the system,
the spot for generating a report will be opened and the command for generating 
a report will be passed, all the while inspecting for errors or missing 
documentation (type: dynamic, functional).			

\end{enumerate}

\subsubsection{Test Area 5}

Applies to all FRs involving creating an account within the system. This currently
covers FR-7 of subsection 9.1 of the SRS \citep{SRS}.

\paragraph{Account Creation Test}

\begin{enumerate}

\item{test-FR7-1\\}

Control: Manual.
					
Initial State: system is active, and spot for account creation is open.
					
Input: command for account creation (button, enter, etc.) alongside account 
user email and password, and the email is not associated with any existing 
account.
					
Output: System notifies user account has been successfully created for 
logging in with and remains active, awaiting further action directives.

Test Case Derivation: The system should be able to assess a set of account 
credentials is not yet within the system and proceed to add this set to the 
set of accounts that can be logged in with.

How test will be performed: Spot for account creation will be open, email and 
password not asssociated with any existing account will be given in the appropriate 
area, and command for account creation will be passed, all the while inspecting 
for any failure mode within the process (type: dynamic, functional).

\item{test-FR7-2\\}

Control: Manual
					
Initial State: system is active, and spot for account creation is open.
					
Input: command for account creation (button, enter, etc.) alongside account 
user email and password, and the email is associated with an existing 
account.
					
Output: System notifies user account was not possible to create for logging 
in with and remains active, awaiting further action directives.

Test Case Derivation: A set of pre-existing account credentials should not be 
possible to create an account with. Otherwise, account creation is arbitrary 
and not truly provided by the system as any set of account credentials are not 
tied to any particular account.

How test will be performed: Spot for account creation will be open, email and 
password asssociated with an existing account will be given in the appropriate 
area, and command for account creation will be passed, all the while inspecting 
for any failure mode within the process (type: dynamic, functional).
					

\end{enumerate}

\subsubsection{Test Area 6}

Applies to all FRs involving logging into account within the system. This 
currently covers FR-8 of subsection 9.1 of the SRS \citep{SRS}.

\paragraph{Account Login Test}

\begin{enumerate}

\item{test-FR8-1\\}

Control: Manual.
					
Initial State: system is active, and spot for account login is open.
					
Input: command for account login (button, enter, etc.) alongside account 
user email and password, and the email is associated with an existing 
account.
					
Output: System notifies user account login was successful and and remains 
active, awaiting further action directives.

Test Case Derivation: The system should be able to validate a set of 
pre-existing account credentials to facilitate login.

How test will be performed: Spot for account login will be open, email and 
password asssociated with existing account will be given in the appropriate 
area, and command for account login will be passed, all the while inspecting 
for any failure mode within the process (type: dynamic, functional).

\item{test-FR8-2\\}

Control: Manual.
					
Initial State: system is active, and spot for account login is open.
					
Input: command for account login (button, enter, etc.) alongside account 
user email and password, and the email is not associated with any existing 
account.
					
Output: System notifies user account login was not successful and remains 
active, awaiting further action directives.

Test Case Derivation: A set of account credentials not associated an existing 
account will fail to allow login as the system should determine there is no 
account to validate against.

How test will be performed: Spot for account login will be open, email and 
password not asssociated with existing account with will be given in the appropriate 
area, and command for account login will be passed, all the while inspecting for 
any failure mode within the process (type: dynamic, functional).
					
\end{enumerate}

\subsubsection{Test Area 7}

Applies to all FRs involving emailing results of plagiarism analysis to users. 
This currently covers FR-9 of subsection 9.1 of the SRS \citep{SRS}.

\paragraph{Result Email Test}

\begin{enumerate}

\item{test-FR9-1\\}

Control: Manual.
					
Initial State: System is active, most recent plagiarism analysis has been 
completed since system activation and its results are available, and spot 
for emailing results is open.
					
Input: command for emailing result (button, enter, etc.) and email to 
receive results.
					
Output: System notifies email has been sent and remains active, awaiting 
further action directives. External to system, the specified email shall
contain a .zip file possessing results of the recent analysis.

Test Case Derivation: The system should possess the ability to condense results
into a .zip file upon demand, and it should proceed to carry out emailing this 
file after it creates it.

How test will be performed: Proceeding a successful analysis run from the system,
the spot for emailing results will be opened and the command for sending email 
will be passed alongside an email for receiving results, all the while inspecting 
for errors in the process. The reception of the .zip file will be assessed at 
the very end once the system is awaiting action directives (type: dynamic, functional).
					

\end{enumerate}

\subsubsection{Test Area 8}

Applies to all FRs involving visualizing plagiarism analysis results provided 
by user in a .zip file. This currently covers FR-10 of subsection 9.1 of the SRS 
\citep{SRS}.

\paragraph{Result Visualization Test}

\begin{enumerate}

\item{test-FR10-1\\}

Control: Manual
					
Initial State: System is active and spot for inserting .zip file containing
results to produce visualization is open.
					
Input: command for visualization (button, enter, etc.) and .zip file containing 
results.
					
Output: System provides viewable visualization that corresponds to the results
in the given .zip file.

Test Case Derivation: System should possess ability to parse .zip file for 
relevant contents and pass it into an internal visualization template that 
can be filled out with what was found before proceeding to transfer it to
the user.

How test will be performed: spot for inserting .zip file to produce 
visualization is open and command for visualization is passed alongside
a .zip file containing results, all the while inspecting for errors or 
missing parts of the visualziation in the process.
\end{enumerate}

\subsection{Tests for Nonfunctional Requirements}

\wss{The nonfunctional requirements for accuracy will likely just reference the
  appropriate functional tests from above.  The test cases should mention
  reporting the relative error for these tests.  Not all projects will
  necessarily have nonfunctional requirements related to accuracy.}

\wss{For some nonfunctional tests, you won't be setting a target threshold for
passing the test, but rather describing the experiment you will do to measure
the quality for different inputs.  For instance, you could measure speed versus
the problem size.  The output of the test isn't pass/fail, but rather a summary
table or graph.}

\wss{Tests related to usability could include conducting a usability test and
  survey.  The survey will be in the Appendix.}

\wss{Static tests, review, inspections, and walkthroughs, will not follow the
format for the tests given below.}

\wss{If you introduce static tests in your plan, you need to provide details.
How will they be done?  In cases like code (or document) walkthroughs, who will
be involved? Be specific.}

\subsubsection{Area of Testing1}
		
\paragraph{Title for Test}

\begin{enumerate}

\item{test-id1\\}

Type: Functional, Dynamic, Manual, Static etc.
					
Initial State: 
					
Input/Condition: 
					
Output/Result: 
					
How test will be performed: 
					
\item{test-id2\\}

Type: Functional, Dynamic, Manual, Static etc.
					
Initial State: 
					
Input: 
					
Output: 
					
How test will be performed: 

\end{enumerate}

\subsubsection{Area of Testing2}

...

\subsection{Traceability Between Test Cases and Requirements}

\wss{Provide a table that shows which test cases are supporting which
  requirements.}

\section{Unit Test Description}
This section will not be filled in until after the MIS document has been completed.
% \wss{This section should not be filled in until after the MIS (detailed design
%   document) has been completed.}

% \wss{Reference your MIS (detailed design document) and explain your overall
% philosophy for test case selection.}  

% \wss{To save space and time, it may be an option to provide less detail in this section.  
% For the unit tests you can potentially layout your testing strategy here.  That is, you 
% can explain how tests will be selected for each module.  For instance, your test building 
% approach could be test cases for each access program, including one test for normal behaviour 
% and as many tests as needed for edge cases.  Rather than create the details of the input 
% and output here, you could point to the unit testing code.  For this to work, you code 
% needs to be well-documented, with meaningful names for all of the tests.}

% \subsection{Unit Testing Scope}

% \wss{What modules are outside of the scope.  If there are modules that are
%   developed by someone else, then you would say here if you aren't planning on
%   verifying them.  There may also be modules that are part of your software, but
%   have a lower priority for verification than others.  If this is the case,
%   explain your rationale for the ranking of module importance.}

% \subsection{Tests for Functional Requirements}

% \wss{Most of the verification will be through automated unit testing.  If
%   appropriate specific modules can be verified by a non-testing based
%   technique.  That can also be documented in this section.}

% \subsubsection{Module 1}

% \wss{Include a blurb here to explain why the subsections below cover the module.
%   References to the MIS would be good.  You will want tests from a black box
%   perspective and from a white box perspective.  Explain to the reader how the
%   tests were selected.}

% \begin{enumerate}

% \item{test-id1\\}

% Type: \wss{Functional, Dynamic, Manual, Automatic, Static etc. Most will
%   be automatic}
					
% Initial State: 
					
% Input: 
					
% Output: \wss{The expected result for the given inputs}

% Test Case Derivation: \wss{Justify the expected value given in the Output field}

% How test will be performed: 
					
% \item{test-id2\\}

% Type: \wss{Functional, Dynamic, Manual, Automatic, Static etc. Most will
%   be automatic}
					
% Initial State: 
					
% Input: 
					
% Output: \wss{The expected result for the given inputs}

% Test Case Derivation: \wss{Justify the expected value given in the Output field}

% How test will be performed: 

% \item{...\\}
    
% \end{enumerate}

% \subsubsection{Module 2}

% ...

% \subsection{Tests for Nonfunctional Requirements}

% \wss{If there is a module that needs to be independently assessed for
%   performance, those test cases can go here.  In some projects, planning for
%   nonfunctional tests of units will not be that relevant.}

% \wss{These tests may involve collecting performance data from previously
%   mentioned functional tests.}

% \subsubsection{Module ?}
		
% \begin{enumerate}

% \item{test-id1\\}

% Type: \wss{Functional, Dynamic, Manual, Automatic, Static etc. Most will
%   be automatic}
					
% Initial State: 
					
% Input/Condition: 
					
% Output/Result: 
					
% How test will be performed: 
					
% \item{test-id2\\}

% Type: Functional, Dynamic, Manual, Static etc.
					
% Initial State: 
					
% Input: 
					
% Output: 
					
% How test will be performed: 

% \end{enumerate}

% \subsubsection{Module ?}

% ...

% \subsection{Traceability Between Test Cases and Modules}

% \wss{Provide evidence that all of the modules have been considered.}
				
\bibliographystyle{plainnat}

\bibliography{../../refs/References}

\newpage

\section{Appendix}

% This is where you can place additional information.

\subsection{Symbolic Parameters}

The definition of the test cases will call for SYMBOLIC\_CONSTANTS.
Their values are defined in this section for easy maintenance.

\subsection{Usability Survey Questions?}

\wss{This is a section that would be appropriate for some projects.}

\newpage{}
\section*{Appendix --- Reflection}

% \wss{This section is not required for CAS 741}

The information in this section will be used to evaluate the team members on the
graduate attribute of Lifelong Learning.

The purpose of reflection questions is to give you a chance to assess your own
learning and that of your group as a whole, and to find ways to improve in the
future. Reflection is an important part of the learning process.  Reflection is
also an essential component of a successful software development process.  

Reflections are most interesting and useful when they're honest, even if the
stories they tell are imperfect. You will be marked based on your depth of
thought and analysis, and not based on the content of the reflections
themselves. Thus, for full marks we encourage you to answer openly and honestly
and to avoid simply writing ``what you think the evaluator wants to hear.''

Please answer the following questions.  Some questions can be answered on the
team level, but where appropriate, each team member should write their own
response:


\begin{enumerate}
  \item What went well while writing this deliverable? 

  While writing this deliverable, our team was able to collaboratively clarify and solidify 
  our understanding of both the functional and non-functional requirements. This process helped us align on 
  specific goals and create comprehensive test plans, which ensures that our testing will effectively verify 
  that all key project requirements are met.

  \item What pain points did you experience during this deliverable, and how
    did you resolve them?

  A significant challenge we faced was the extensive amount of non-functional requirements (NFRs), 
  with over 30 NFRs to address. Each NFR required us to develop a detailed test plan, specifying 
  factors like type (e.g., functional, dynamic, manual), initial state, input/conditions, expected 
  output/results, and test method. While the template helped streamline our process, the sheer volume 
  of NFRs meant the documentation grew quickly, and managing this without sacrificing detail was challenging. 
  We prioritized efficiency by dividing NFRs among team members and holding review sessions to ensure consistent 
  quality and adherence to our testing criteria. This allowed us to maintain clarity without being overwhelmed by 
  the documentation demands.
  One of the main challenges we faced was the large amount of 
  \item What knowledge and skills will the team collectively need to acquire to
  successfully complete the verification and validation of your project?
  Examples of possible knowledge and skills include dynamic testing knowledge,
  static testing knowledge, specific tool usage, Valgrind etc.  You should look to
  identify at least one item for each team member.
  \begin{itemize}
    \item \textbf{Dynamic testing knowledge}: 
    \item \textbf{Static testing knowledge}:
    \item \textbf{Automated testing tools}: 
  \end{itemize}

  \item For each of the knowledge areas and skills identified in the previous
  question, what are at least two approaches to acquiring the knowledge or
  mastering the skill?  Of the identified approaches, which will each team
  member pursue, and why did they make this choice?
  \begin{itemize}
    \item \textbf{Lucas Chen:} 
    \item \textbf{Dennis Fong:}
    \item \textbf{Julian Cecchini:} 
    \item \textbf{Mohammad Mohsin Khan:}
    \item \textbf{Luigi Quattrociocchi:}
\end{itemize}
\end{enumerate}

\end{document}
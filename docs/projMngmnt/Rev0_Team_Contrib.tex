\documentclass{article}

\usepackage{float}
\restylefloat{table}

\usepackage{booktabs}

\title{Team Contributions: Rev 0\\\progname}

\author{\authname}

\date{}

%% Comments

\usepackage{color}

\newif\ifcomments\commentstrue %displays comments
%\newif\ifcomments\commentsfalse %so that comments do not display

\ifcomments
\newcommand{\authornote}[3]{\textcolor{#1}{[#3 ---#2]}}
\newcommand{\todo}[1]{\textcolor{red}{[TODO: #1]}}
\else
\newcommand{\authornote}[3]{}
\newcommand{\todo}[1]{}
\fi

\newcommand{\wss}[1]{\authornote{blue}{SS}{#1}} 
\newcommand{\plt}[1]{\authornote{magenta}{TPLT}{#1}} %For explanation of the template
\newcommand{\an}[1]{\authornote{cyan}{Author}{#1}}

%% Common Parts

\newcommand{\progname}{Software Engineering} % PUT YOUR PROGRAM NAME HERE
\newcommand{\authname}{Team 2, SyntaxSentinals
\\ Lucas Chen
\\ Dennis Fong
\\ Mohammad Mohsin Khan
\\ Julian Cecchini
\\ Luigi Quattrociocchi} % AUTHOR NAMES                  

\usepackage{hyperref}
    \hypersetup{colorlinks=true, linkcolor=blue, citecolor=blue, filecolor=blue,
                urlcolor=blue, unicode=false}
    \urlstyle{same}
                                


\begin{document}

\maketitle

This document summarizes the contributions of each team member for the Rev 0
Demo.  The time period of interest is the time between the POC demo and the Rev
0 demo.

\section{Demo Plans}

% \wss{What will you be demonstrating}

As stated in the module guide, we are intended to have all modules complete but
the Flagging Module, Report Results Module, and Email Sending Module. Those modules
are to be collectively completed by February 14th, after our Rev 0 Demo on February 3rd. 
Also, the Results Upload Module hinges on the Email Sending Module (to make results available to 
submit in the first place), so it's functionality will not be fully realized in this
demo.

Therefore, the following components will be done in their stated order as part of the Rev 0 Demo:

\begin{description}
    \item[1 User Authentication:] The team will navigate the UI to a screen where they can 
    register an account and login using it. This account will be non-trivial and this will 
    be demonstrated by attempting to login with erroneous credentials after registering and 
    logging in with the first account.

    \item[2 Code Upload:] The team will navigate the UI to a screen where snippets of code 
    can be uploaded. It will be demonstrated that the UI can take in code snippets in ZIP file 
    format. These code snippets will be utilized for analysis in a later part of the demonstration.

    \item[3 Results Upload:] The team will show a spot where report files arising from the email 
    module will be processed and its ability to piece together basic result information from a text file,
    demonstrating its future functionality.

    \item[4 Threshold Adjustment:] The team will navigate the UI to a screen where thresholds passed
    for analysis can be adjusted by the user. It will be demonstrated that through user input 
    of keyboard entries or mouse clicks, the threshold to be passed to the backend will change.

    \item[5 Analysis with Basic Visualization:] This will be the final part of the demonstration where 
    the team will navigate the UI to the current spot for initiating plagiarism analysis. It will be 
    demonstrated that the user can initiate the analysis through mouse clicks, provided that code has 
    been uploaded as it was earlier on in the demonstration. Seeing that the dedicated code visualization 
    module is not scheduled for completion, the expected output will be the UI providing a block of text and 
    possibly a basic graph visual indicating what the relations the plagiarism detection models found 
    between code snippet input pairs. The text should contain n choose 2 scores as stated in the requirements.
    It will also be made evident that the provided output utilized code snippets uploaded within the demo
    as well as the current threshold score selected by the user. After verification of each of these aspects,
    the demo will conclude.

\end{description}

This leaves our demonstration closer to a minimum viable product (MVP) but not quite thanks to
the threshold adjustment and some more refined model processing in the back end. Further adjustments 
and refinements.


\section{Team Meeting Attendance}

\begin{table}[H]
\centering
\begin{tabular}{ll}
\toprule
\textbf{Student} & \textbf{Meetings}\\
\midrule
Total & 1\\
Mohammad Mohsin Khan & 1\\
Luigi Quattrociocchi & 1\\
Julian Cecchini & 1\\
Dennis Fong & 1\\
Lucas Chen & 1\\
\bottomrule
\end{tabular}
\end{table}

Overall, the team has had 1 meeting since the POC demo to discuss the project. This number may seem low, but the team has been in constant communication through other means such as Discord and GitHub.

\section{Supervisor/Stakeholder Meeting Attendance}

\begin{table}[H]
\centering
\begin{tabular}{ll}
\toprule
\textbf{Student} & \textbf{Meetings}\\
\midrule
Total & 0\\
Mohammad Mohsin Khan & 0\\
Lucas Chen & 0\\
Dennis Fong & 0\\
Julian Cecchini & 0\\
Luigi Quattrociocchi & 0\\
\bottomrule
\end{tabular}
\end{table}

We haven't been able to meet with our supervisor since our POC demo since he is on a sabatical for the winter term but he was satisfied with the team's progress during the POC demo.

\section{Lecture Attendance}

\begin{table}[H]
\centering
\begin{tabular}{ll}
\toprule
\textbf{Student} & \textbf{Lectures}\\
\midrule
Total & 2\\
Dennis Fong & 1\\
Lucas Chen & 1\\
Luigi Quattrociocchi & 2\\
Mohammed Mohsin Khan & 2\\
Julian Cecchini & 1\\
\bottomrule
\end{tabular}
\end{table}

\section{TA Document Discussion Attendance}

\begin{table}[H]
\centering
\begin{tabular}{ll}
\toprule
\textbf{Student} & \textbf{Lectures}\\
\midrule
Total & 1\\
Mohammad Mohsin Khan & 1\\
Lucas Chen & 1\\
Dennis Fong & 1\\
Luigi Quattrociocchi & 1\\
Julian Cecchini & 1\\
\bottomrule
\end{tabular}
\end{table}

\section{Commits}

% \wss{For each team member how many commits to the main branch have been made
% over the time period of interest.  The total is the total number of commits for
% the entire team since the beginning of the term.  The percentage is the
% percentage of the total commits made by each team member.}

\begin{table}[H]
\centering
\begin{tabular}{lll}
\toprule
\textbf{Student} & \textbf{Commits} & \textbf{Percent}\\
\midrule
Total & 183 & 100\% \\
Mohammad Mohsin Khan & 31 & 16.9\% \\
Lucas Chen & 59 & 32.3\% \\
Dennis Fong & 18 & 10.2\% \\
Julian Cecchini & 27 & 14.8\% \\
Luigi Quattrociocchi & 47 & 25.7\% \\
\bottomrule
\end{tabular}
\end{table}

% \wss{If needed, an explanation for the counts can be provided here.  For
% instance, if a team member has more commits to unmerged branches, these numbers
% can be provided here.  If multiple people contribute to a commit, git allows for
% multi-author commits.}

\section{Issue Tracker}

% \wss{For each team member how many issues have they authored (including open and
% closed issues (O+C)) and how many have they been assigned (only counting closed
% issues (C only)) over the time period of interest.}

\begin{table}[H]
\centering
\begin{tabular}{lll}
\toprule
\textbf{Student} & \textbf{Authored (O+C)} & \textbf{Assigned (C only)}\\
\midrule
Dennis Fong & 0 & 5 \\
Mohammad Mohsin Khan & 3 & 15 \\
Luigi Quattrociocchi & 1 & 7 \\
Lucas Chen & 9 & 14 \\
Julian Cecchini & 0 & 11 \\
\bottomrule
\end{tabular}
\end{table}

% \wss{If needed, an explanation for the counts can be provided here.}

\section{CICD}

In our project, we use CI/CD pipelines to automate various tasks and streamline our workflow. Specifically:

\begin{itemize}
    \item \textbf{GitHub Actions for LaTeX Compilation}: We have a GitHub Actions workflow that automatically compiles our LaTeX documents into PDFs whenever changes are pushed, ensuring that our documentation is always up to date.
    \item \textbf{Frontend Deployment}: We use CI/CD to automatically deploy our frontend application whenever changes are merged into the \texttt{main} branch. This ensures that the latest version is always live without manual intervention.
    \item \textbf{Branch Deployments}: For feature branches, we have CI/CD workflows that deploy changes to a staging environment, allowing us to test new features before merging them into \texttt{main}.
\end{itemize}

% \wss{Say how CICD is used in your project}


% \wss{If your team has additional metrics of productivity, please feel free to
% add them to this report.}

\end{document}
\documentclass{article}

\usepackage{float}
\restylefloat{table}

\usepackage{booktabs}

\title{Team Contributions: Final\\\progname}

\author{\authname}

\date{}

%% Comments

\usepackage{color}

\newif\ifcomments\commentstrue %displays comments
%\newif\ifcomments\commentsfalse %so that comments do not display

\ifcomments
\newcommand{\authornote}[3]{\textcolor{#1}{[#3 ---#2]}}
\newcommand{\todo}[1]{\textcolor{red}{[TODO: #1]}}
\else
\newcommand{\authornote}[3]{}
\newcommand{\todo}[1]{}
\fi

\newcommand{\wss}[1]{\authornote{blue}{SS}{#1}} 
\newcommand{\plt}[1]{\authornote{magenta}{TPLT}{#1}} %For explanation of the template
\newcommand{\an}[1]{\authornote{cyan}{Author}{#1}}

%% Common Parts

\newcommand{\progname}{Software Engineering} % PUT YOUR PROGRAM NAME HERE
\newcommand{\authname}{Team 2, SyntaxSentinals
\\ Lucas Chen
\\ Dennis Fong
\\ Mohammad Mohsin Khan
\\ Julian Cecchini
\\ Luigi Quattrociocchi} % AUTHOR NAMES                  

\usepackage{hyperref}
    \hypersetup{colorlinks=true, linkcolor=blue, citecolor=blue, filecolor=blue,
                urlcolor=blue, unicode=false}
    \urlstyle{same}
                                


\begin{document}

\maketitle

This document summarizes the contributions of each team member for the final
demonstration and documentation.  The time period of interest is the time
between Rev 0 and the Final documentation; the contributions prior to Rev0 are
NOT included.

\section{Team Meeting Attendance}

% \wss{For each team member how many team meetings have they attended over the
% time period of interest.  This number should be determined from the meeting
% issues in the team's repo.  The first entry in the table should be the total
% number of team meetings held by the team.}

\begin{table}[H]
\centering
\begin{tabular}{ll}
\toprule
\textbf{Student} & \textbf{Meetings}\\
\midrule
Total & 3\\
Luigi Quattrociocchi & 3\\
Dennis Fong & 3\\
Mohammed Mohsin Khan & 3\\
Lucas Chen & 3\\
Julian Cecchini & 3\\
\end{tabular}
\end{table}

% \wss{If needed, an explanation for the counts can be provided here.}

\section{Supervisor/Stakeholder Meeting Attendance}

% \wss{For each team member how many supervisor/stakeholder team meetings have
% they attended over the time period of interest.  This number should be determined
% from the supervisor meeting issues in the team's repo.  The first entry in the
% table should be the total number of supervisor and team meetings held by the
% team.  If there is no supervisor, there will usually be meetings with
% stakeholders (potential users) that can serve a similar purpose.}

\noindent \textbf{Supervisor's Name: } Dr. Hassan Ashtiani

\begin{table}[H]
\centering
\begin{tabular}{ll}
\toprule
\textbf{Student} & \textbf{Meetings}\\
\midrule
Total & 0\\
Mohammad Mohsin Khan & 0\\
Luigi Quattrociocchi & 0\\
Julian Cecchini & 0\\
Dennis Fong & 0\\
Lucas Chen & 0\\
\bottomrule
\end{tabular}
\end{table}

Dr. Hassan Ashtiani left for sabbatical during the winter semester as discussed with 
Dr. Smith. Therefore, we have not had any follow-ups since our initial meetings in the 
fall semester. However, it is worthwhile to note we have not been significantly derailed 
in our direction and have followed through with pointers he provided (such as tinkering 
with online datasets to extend what the model can work with).

\section{Lecture Attendance}

% \wss{For each team member how many lectures have they attended over the time
% period of interest.  This number should be determined from the lecture issues in
% the team's repo. You can find the number of lectures in the time period of
% interest by looking at the
% \href{https://calendar.google.com/calendar/u/0/embed?src=rnboqiaki1k2la7rpu3bn0um58@group.calendar.google.com&ctz=America/Toronto}
% {Google calendar} for the capstone course.}

\begin{table}[H]
\centering
\begin{tabular}{ll}
\toprule
\textbf{Student} & \textbf{Lectures}\\
\midrule
Total & 1\\
Mohammad Mohsin Khan & 0\\
Luigi Quattrociocchi & 0\\
Julian Cecchini & 0\\
Dennis Fong & 0\\
Lucas Chen & 0\\
\bottomrule
\end{tabular}
\end{table}

Unfortunately, all members happened to be busy during the final lecture on March 7th.

\section{TA Document Discussion Attendance}

% \wss{For each team member how many of the informal document discussion meetings
% with the TA were attended over the time period of interest.}

\noindent \textbf{TA's Name: } Lucas Dutton

\begin{table}[H]
\centering
\begin{tabular}{ll}
\toprule
\textbf{Student} & \textbf{Lectures}\\
\midrule
Total & 1\\
Mohammad Mohsin Khan & 1\\
Luigi Quattrociocchi & 1\\
Julian Cecchini & 1\\
Dennis Fong & 1\\
Lucas Chen & 1\\
\bottomrule
\end{tabular}
\end{table}

% \wss{If needed, an explanation for the attendance can be provided here.}

\section{Commits}

% \wss{For each team member how many commits to the main branch have been made
% over the time period of interest.  The total is the total number of commits for
% the entire team since the beginning of the term.  The percentage is the
% percentage of the total commits made by each team member.}

\begin{table}[H]
\centering
\begin{tabular}{lll}
\toprule
\textbf{Student} & \textbf{Commits} & \textbf{Percent}\\
\midrule
Total & Num & 100\% \\
Mohammad Mohsin Khan & 41 & 18.98\% \\
Luigi Quattrociocchi & 52 & 24.07\% \\
Julian Cecchini & 32 & 14.81\% \\
Dennis Fong & 22 & 10.19\% \\
Lucas Chen & 69 & 31.94\% \\
\bottomrule
\end{tabular}
\end{table}

% \wss{If needed, an explanation for the counts can be provided here.  For
% instance, if a team member has more commits to unmerged branches, these numbers
% can be provided here.  If multiple people contribute to a commit, git allows for
% multi-author commits.}

\section{Issue Tracker}

% \wss{For each team member how many issues have they authored (including open and
% closed issues (O+C)) and how many have they been assigned (only counting closed
% issues (C only)) over the time period of interest.}

\begin{table}[H]
\centering
\begin{tabular}{lll}
\toprule
\textbf{Student} & \textbf{Authored (O+C)} & \textbf{Assigned (C only)}\\
\midrule
Mohammad Mohsin Khan & 14 & 10 \\
Luigi Quattrociocchi & 0 & 5 \\
Julian Cecchini & 0 & 5 \\
Dennis Fong & 0 & 6 \\
Lucas Chen & 0 & 4 \\
\bottomrule
\end{tabular}
\end{table}

% \wss{If needed, an explanation for the counts can be provided here.}
Although most of us did not author any issues or close a lot of them, we all contributed
to the issues in some way. A better metric would be to count the number of
commits that were made by each memeber that was summarized in the previous
section.

\section{Team Charter Trigger Items}

Most of the following items in the team charter were followed during the project:
\begin{enumerate}
    \item Attendance
    \begin{enumerate}
        \item Expectations
        \item Acceptable Excuse
        \item In Case of Emergency
    \end{enumerate}
    \item Accountability and Teamwork
    \begin{enumerate}
        \item Quality
        \item Attitude
        \item Stay on Track
        \item Team Building
        \item Decision Making
    \end{enumerate}
\end{enumerate}

Regarding accountability and teamwork: our attitude and tracking both
faltered somewhat over the course of the project. The staying on track aspect
was expected to fall behind at least a little bit, as we are not perfect and
our development sometimes went off-script. Many times we would develop features
and change functionality without having created issues for traceability, for
various reasons including laziness or impromptu changes. Also, other things
happened to interfere with the timeline, which also contributed to the lack of
tracking and degradation of attitude. The team attitude became more strenuous
over time because of disagreements in direction, and stress caused by constant
aiming to hit deadlines.

Going forward, we planned to and continue to plan to resolve these violations
by keeping open and honest communication, as well as frequent meeting and
discussion about the direction and goals of the project.

\section{Additional Productivity Metrics}

Some contributions to the project were unseen by the repo, because there was not
necessarily code to commit for it. This includes matters such as creating data 
in the form of code snippets, researching and testing algorithms from online sources
against our own before deciding on implementation, or labelling data to train our 
model with. Towards this end, we decided to make a metric to track all this 
in the form of data procurement/research quantified by hours.

\begin{table}[H]
    \centering
    \begin{tabular}{ll}
    \toprule
    \textbf{Student} & \textbf{Hours Researching/Procuring}\\
    \midrule
    Total & \\
    Mohammad Mohsin Khan & 4\\
    Luigi Quattrociocchi & 9\\
    Julian Cecchini & 11\\
    Dennis Fong & 13\\
    Lucas Chen & 4\\
    \bottomrule
    \end{tabular}
\end{table}
    


\end{document}
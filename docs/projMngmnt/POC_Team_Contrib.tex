\documentclass{article}

\usepackage{float}
\restylefloat{table}

\usepackage{booktabs}

\title{Team Contributions: POC\\\progname}

\author{\authname}

\date{}

%% Comments

\usepackage{color}

\newif\ifcomments\commentstrue %displays comments
%\newif\ifcomments\commentsfalse %so that comments do not display

\ifcomments
\newcommand{\authornote}[3]{\textcolor{#1}{[#3 ---#2]}}
\newcommand{\todo}[1]{\textcolor{red}{[TODO: #1]}}
\else
\newcommand{\authornote}[3]{}
\newcommand{\todo}[1]{}
\fi

\newcommand{\wss}[1]{\authornote{blue}{SS}{#1}} 
\newcommand{\plt}[1]{\authornote{magenta}{TPLT}{#1}} %For explanation of the template
\newcommand{\an}[1]{\authornote{cyan}{Author}{#1}}

%% Common Parts

\newcommand{\progname}{Software Engineering} % PUT YOUR PROGRAM NAME HERE
\newcommand{\authname}{Team 2, SyntaxSentinals
\\ Lucas Chen
\\ Dennis Fong
\\ Mohammad Mohsin Khan
\\ Julian Cecchini
\\ Luigi Quattrociocchi} % AUTHOR NAMES                  

\usepackage{hyperref}
    \hypersetup{colorlinks=true, linkcolor=blue, citecolor=blue, filecolor=blue,
                urlcolor=blue, unicode=false}
    \urlstyle{same}
                                


\begin{document}

\maketitle

This document summarizes the contributions of each team member up to the POC
Demo.  The time period of interest is the time between the beginning of the term
and the POC demo.

\section{Demo Plans}

\wss{What will you be demonstrating}

\section{Team Meeting Attendance}

% \wss{For each team member how many team meetings have they attended over the
% time period of interest.  This number should be determined from the meeting
% issues in the team's repo.  The first entry in the table should be the total
% number of team meetings held by the team.}

\begin{table}[H]
\centering
\begin{tabular}{ll}
\toprule
\textbf{Student} & \textbf{Meetings}\\
\midrule
Total & 3\\
Mohammad Mohsin Khan & 3\\
Lucas Chen & 3\\
Dennis Fong & 3\\
Julian Cecchini & 3\\
Luigi Quattrociocchi & 3\\
\bottomrule
\end{tabular}
\end{table}

Our team has met on more than just 3 occasions, however we have only
had this many official meetings which we recorded. Much of the time
we work together and brainstorm ideas is not recorded explicitly.

\section{Supervisor/Stakeholder Meeting Attendance}

\wss{For each team member how many supervisor/stakeholder team meetings have
they attended over the time period of interest.  This number should be determined
from the supervisor meeting issues in the team's repo.  The first entry in the
table should be the total number of supervisor and team meetings held by the
team.  If there is no supervisor, there will usually be meetings with
stakeholders (potential users) that can serve a similar purpose.}

\begin{table}[H]
\centering
\begin{tabular}{ll}
\toprule
\textbf{Student} & \textbf{Meetings}\\
\midrule
Total & Num\\
Mohammad Mohsin Khan & Num\\
Lucas Chen & Num\\
Dennis Fong & Num\\
Julian Cecchini & Num\\
Luigi Quattrociocchi & Num\\
\bottomrule
\end{tabular}
\end{table}

\wss{If needed, an explanation for the counts can be provided here.}

\section{Lecture Attendance}

\wss{For each team member how many lectures have they attended over the time
period of interest.  This number should be determined from the lecture issues in
the team's repo.  The first entry in the table should be the total number of
lectures since the beginning of the term.}

\begin{table}[H]
\centering
\begin{tabular}{ll}
\toprule
\textbf{Student} & \textbf{Lectures}\\
\midrule
Total & Num\\
Mohammad Mohsin Khan & Num\\
Lucas Chen & Num\\
Dennis Fong & Num\\
Julian Cecchini & Num\\
Luigi Quattrociocchi & Num\\
\bottomrule
\end{tabular}
\end{table}

\wss{If needed, an explanation for the lecture attendance can be provided here.}

\section{TA Document Discussion Attendance}

\begin{table}[H]
\centering
\begin{tabular}{ll}
\toprule
\textbf{Student} & \textbf{Lectures}\\
\midrule
Total & 3\\
Mohammad Mohsin Khan & 3\\
Lucas Chen & 3\\
Dennis Fong & 3\\
Julian Cecchini & 3\\
Luigi Quattrociocchi & 3\\
\bottomrule
\end{tabular}
\end{table}

\section{Commits}

\wss{For each team member how many commits to the main branch have been made
over the time period of interest.  The total is the total number of commits for
the entire team since the beginning of the term.  The percentage is the
percentage of the total commits made by each team member.}

\begin{table}[H]
\centering
\begin{tabular}{lll}
\toprule
\textbf{Student} & \textbf{Commits} & \textbf{Percent}\\
\midrule
Total & 118 & 100\% \\
Mohammad Mohsin Khan & 18 & 15.3\% \\
Lucas Chen & 36 & 30.5\% \\
Dennis Fong & 15 & 12.7\% \\
Julian Cecchini & 15 & 12.7\% \\
Luigi Quattrociocchi & 34 & 28.8\% \\
\bottomrule
\end{tabular}
\end{table}

\wss{If needed, an explanation for the counts can be provided here.  For
instance, if a team member has more commits to unmerged branches, these numbers
can be provided here.  If multiple people contribute to a commit, git allows for
multi-author commits.}

\section{Issue Tracker}

\begin{table}[H]
\centering
\begin{tabular}{lll}
\toprule
\textbf{Student} & \textbf{Authored (O+C)} & \textbf{Assigned (C only)}\\
\midrule
Mohammad Mohsin Khan & 40 & 36\\
Lucas Chen & 29 & 26\\
Dennis Fong & 20 & 18\\
Julian Cecchini & 19 & 16\\
Luigi Quattrociocchi & 28 & 25\\
\bottomrule
\end{tabular}
\end{table}

\section{CICD}

The project will use continuous integration and continuous deployment (CICD) to run tests and deploy the software.
The steps in the CICD pipeline are as follows:
\begin{enumerate}
  \item Developer creates PR and CICD pipeline will trigger phase 0.
  \item Phase 0 will run build and check if the build is successful.
  \item If the build is successful, reviewer will review the PR.
  \item If the reviewer approves the PR, the CICD pipeline will trigger phase 1.
  \item Phase 1 will run tests and check if the tests are successful.
  \item If the tests are successful, the CICD pipeline will trigger phase 2.
  \item Phase 2 will merge and deploy the software.
\end{enumerate}


% \wss{Say how CICD will be used in your project}

% \wss{If your team has additional metrics of productivity, please feel free to
% add them to this report.}

\end{document}
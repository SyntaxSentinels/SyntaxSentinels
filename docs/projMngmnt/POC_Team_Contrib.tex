\documentclass{article}

\usepackage{float}
\restylefloat{table}

\usepackage{booktabs}

\title{Team Contributions: POC\\\progname}

\author{\authname}

\date{}

%% Comments

\usepackage{color}

\newif\ifcomments\commentstrue %displays comments
%\newif\ifcomments\commentsfalse %so that comments do not display

\ifcomments
\newcommand{\authornote}[3]{\textcolor{#1}{[#3 ---#2]}}
\newcommand{\todo}[1]{\textcolor{red}{[TODO: #1]}}
\else
\newcommand{\authornote}[3]{}
\newcommand{\todo}[1]{}
\fi

\newcommand{\wss}[1]{\authornote{blue}{SS}{#1}} 
\newcommand{\plt}[1]{\authornote{magenta}{TPLT}{#1}} %For explanation of the template
\newcommand{\an}[1]{\authornote{cyan}{Author}{#1}}

%% Common Parts

\newcommand{\progname}{Software Engineering} % PUT YOUR PROGRAM NAME HERE
\newcommand{\authname}{Team 2, SyntaxSentinals
\\ Lucas Chen
\\ Dennis Fong
\\ Mohammad Mohsin Khan
\\ Julian Cecchini
\\ Luigi Quattrociocchi} % AUTHOR NAMES                  

\usepackage{hyperref}
    \hypersetup{colorlinks=true, linkcolor=blue, citecolor=blue, filecolor=blue,
                urlcolor=blue, unicode=false}
    \urlstyle{same}
                                


\begin{document}

\maketitle

This document summarizes the contributions of each team member up to the POC
Demo.  The time period of interest is the time between the beginning of the term
and the POC demo.

\section{Demo Plans}

% \wss{What will you be demonstrating}
In our proof of concept demonstration, our team will showcase our approach
to code plagiarism detection using NLP by comparing two code samples and generating a similarity score.

In our demonstration, we will invoke a command line application. The application will take
in two source code files in either Python or Java as input and will return a similarity score
as a number from 0.0 to 1.0. A score of 0.0 indicates that the source files have no semantic similarity,
and a score of 1.0 means that the two source files are entirely semantically similar.
We will invoke this application twice, first with two source files that do obviously different things.
These files will have a low similarity score, close to 0.0. The second invocation will be with two
source files that do the same thing but differ syntactically. These files will have a high similarity score,
close to 1.0. The produced similarity scores should make sense when validated by us as human reviewers.

This demonstration aims to prove that the similarity score produced by our NLP model is a good metric
for determining if two pieces of code are plagiarised or not.


\section{Team Meeting Attendance}

% \wss{For each team member how many team meetings have they attended over the
% time period of interest.  This number should be determined from the meeting
% issues in the team's repo.  The first entry in the table should be the total
% number of team meetings held by the team.}

\begin{table}[H]
\centering
\begin{tabular}{ll}
\toprule
\textbf{Student} & \textbf{Meetings}\\
\midrule
Total & 3\\
Mohammad Mohsin Khan & 3\\
Lucas Chen & 3\\
Dennis Fong & 3\\
Julian Cecchini & 3\\
Luigi Quattrociocchi & 3\\
\bottomrule
\end{tabular}
\end{table}

Our team has met on more than just 3 occasions, however, we have only
had this many official meetings which we recorded. Much of the time
we work together and brainstorm ideas is not recorded explicitly.

\section{Supervisor/Stakeholder Meeting Attendance}

% \wss{For each team member how many supervisor/stakeholder team meetings have
% they attended over the time period of interest.  This number should be determined
% from the supervisor meeting issues in the team's repo.  The first entry in the
% table should be the total number of supervisor and team meetings held by the
% team.  If there is no supervisor, there will usually be meetings with
% stakeholders (potential users) that can serve a similar purpose.}

\begin{table}[H]
\centering
\begin{tabular}{ll}
\toprule
\textbf{Student} & \textbf{Meetings}\\
\midrule
Total & 2\\
Mohammad Mohsin Khan & 0\\
Lucas Chen & 0\\
Dennis Fong & 1\\
Julian Cecchini & 2\\
Luigi Quattrociocchi & 0\\
\bottomrule
\end{tabular}
\end{table}

% \wss{If needed, an explanation for the counts can be provided here.}

The two meetings consisted of a supervisor meeting and stakeholder meeting. 

Our supervisor, Dr. Hassan Ashtiani, agreed to supervise our project on special 
terms with his fairly restrictive schedule, which Dr. Smith is aware of. After 
an initial meeting with him for project scope, our group decided it would be wiser 
to have our next meeting with him for the POC phase seeing his expertise in ML 
implementation would be more helpful for starting to code instead of refactoring 
requirements. Therefore, our next meeting with him has not happened yet but will 
be happening very soon. The initial scope meeting only had part of the group that
held experience with ML algorithms and techniques to better judge with Hassan where 
our domain knowledge was lacking and what parts of the project we would need to rely 
on his knowledge for.

The initial skakeholder meeting felt fairly encompassing of all our concerns/
questions. It was held with Dr. Sebastian Mosser, a member of the CAS department 
who has explored plagiarism detection and NLPs for use in his own classroom before. 
He made clear difficulties he faced with it and ways we could potentially do better 
with our own. He also expressed interest in the potential of the project and made clear
it could hold use for other professors. Therefore, we did not feel inclined to hold 
another stakeholder meeting until we have usability or requirement reviews for later 
SRS revisions or code implementations where we would like to affirm if our direction 
is still correct. The meeting was held only between Dr. Mosser and a member of the 
group who had previously TA'd for him as they were able to reach out to Mosser and 
leverage their familiarity to arrange a meeting scenario Mosser was happy with.
% \wss{For each team member how many lectures have they attended over the time
% period of interest.  This number should be determined from the lecture issues in
% the team's repo.  The first entry in the table should be the total number of
% lectures since the beginning of the term.}

\section{Lecture Attendance}

\begin{table}[H]
\centering
\begin{tabular}{ll}
\toprule
\textbf{Student} & \textbf{Lectures}\\
\midrule
Total & 12\\
Mohammad Mohsin Khan & 5\\
Lucas Chen & 5\\
Dennis Fong & 5\\
Julian Cecchini & 5\\
Luigi Quattrociocchi & 5\\
\bottomrule
\end{tabular}
\end{table}

Every team member has attended every lecture since we started tracking
this metric by recording lecture attendance. In the above table, we each
have only attended 5 lectures because we didn't track this until a couple
weeks into the semester (September 17th is our first recorded lecture).
In fact, all team members have attended every lecture, except for the
VnV plan lecture on October 23rd, which no one attended because we all
had a midterm on that day.

\section{TA Document Discussion Attendance}

\begin{table}[H]
\centering
\begin{tabular}{ll}
\toprule
\textbf{Student} & \textbf{Lectures}\\
\midrule
Total & 3\\
Mohammad Mohsin Khan & 3\\
Lucas Chen & 3\\
Dennis Fong & 3\\
Julian Cecchini & 3\\
Luigi Quattrociocchi & 3\\
\bottomrule
\end{tabular}
\end{table}

\section{Commits}

% \wss{For each team member how many commits to the main branch have been made
% over the time period of interest.  The total is the total number of commits for
% the entire team since the beginning of the term.  The percentage is the
% percentage of the total commits made by each team member.}

\begin{table}[H]
\centering
\begin{tabular}{lll}
\toprule
\textbf{Student} & \textbf{Commits} & \textbf{Percent}\\
\midrule
Total & 130 & 100\% \\
Mohammad Mohsin Khan & 20 & 15.4\% \\
Lucas Chen & 36 & 27.7\% \\
Dennis Fong & 15 & 11.5\% \\
Julian Cecchini & 18 & 13.8\% \\
Luigi Quattrociocchi & 41 & 31.5\% \\
\bottomrule
\end{tabular}
\end{table}

Part of the reason for Luigi's high number of commits is due to him fixing
minor issues (chores) such as moving files or changing issue templates. He was
also committing to other people's branches to resolve merge conflicts on their behalf,
and leaving review comments with suggested changes, which makes him a co-author of
their commit. Every member co-authored commits to some extent, which is why the total
number of commits is approximately 30\% higher than the number of commits to the main branch.

% \wss{If needed, an explanation for the counts can be provided here.  For
% instance, if a team member has more commits to unmerged branches, these numbers
% can be provided here.  If multiple people contribute to a commit, git allows for
% multi-author commits.}

\section{Issue Tracker}

\begin{table}[H]
\centering
\begin{tabular}{lll}
\toprule
\textbf{Student} & \textbf{Authored (O+C)} & \textbf{Assigned (C only)}\\
\midrule
Mohammad Mohsin Khan & 6 & 40\\
Lucas Chen & 83 & 29\\
Dennis Fong & 0 & 21\\
Julian Cecchini & 2 & 20\\
Luigi Quattrociocchi & 5 & 31\\
\bottomrule
\end{tabular}
\end{table}

Until now, Lucas has been taking the initiative to create GitHub issues for
subtasks within each deliverable, and assign them to the rest of the team.
This is why he has the majority of authored issues on the repository. The
issues created by other team members were either to document meetings or
to record lecture attendance.

Further, Mohsin's assigned commits include those for team and TA meetings, which
are assigned to him because he is designated to take notes and meeting minutes.

\section{CICD}

The project will use continuous integration and continuous deployment (CICD) to run tests and deploy the software.
The steps in the CICD pipeline are as follows:
\begin{enumerate}
  \item Developer creates PR and CICD pipeline will trigger phase 0.
  \item Phase 0 will run build and check if the build is successful.
  \item If the build is successful, reviewer will review the PR.
  \item If the reviewer approves the PR, the CICD pipeline will trigger phase 1.
  \item Phase 1 will run tests and check if the tests are successful.
  \item If the tests are successful, the CICD pipeline will trigger phase 2.
  \item Phase 2 will merge and deploy the software.
\end{enumerate}


% \wss{Say how CICD will be used in your project}

% \wss{If your team has additional metrics of productivity, please feel free to
% add them to this report.}

\end{document}
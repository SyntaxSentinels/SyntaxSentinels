\documentclass[12pt, titlepage]{article}

\usepackage{booktabs}
\usepackage{tabularx}
\usepackage{hyperref}
\usepackage{longtable}

\hypersetup{
    colorlinks,
    citecolor=black,
    filecolor=black,
    linkcolor=red,
    urlcolor=blue
}
\usepackage[round]{natbib}

%% Comments

\usepackage{color}

\newif\ifcomments\commentstrue %displays comments
%\newif\ifcomments\commentsfalse %so that comments do not display

\ifcomments
\newcommand{\authornote}[3]{\textcolor{#1}{[#3 ---#2]}}
\newcommand{\todo}[1]{\textcolor{red}{[TODO: #1]}}
\else
\newcommand{\authornote}[3]{}
\newcommand{\todo}[1]{}
\fi

\newcommand{\wss}[1]{\authornote{blue}{SS}{#1}} 
\newcommand{\plt}[1]{\authornote{magenta}{TPLT}{#1}} %For explanation of the template
\newcommand{\an}[1]{\authornote{cyan}{Author}{#1}}

%% Common Parts

\newcommand{\progname}{Software Engineering} % PUT YOUR PROGRAM NAME HERE
\newcommand{\authname}{Team 2, SyntaxSentinals
\\ Lucas Chen
\\ Dennis Fong
\\ Mohammad Mohsin Khan
\\ Julian Cecchini
\\ Luigi Quattrociocchi} % AUTHOR NAMES                  

\usepackage{hyperref}
    \hypersetup{colorlinks=true, linkcolor=blue, citecolor=blue, filecolor=blue,
                urlcolor=blue, unicode=false}
    \urlstyle{same}
                                


\begin{document}

\title{Verification and Validation Report: \progname} 
\author{\authname}
\date{\today}
	
\maketitle

\pagenumbering{roman}

\section{Revision History}

\begin{tabularx}{\textwidth}{p{3cm}p{2cm}X}
\toprule {\bf Date} & {\bf Version} & {\bf Notes}\\
\midrule
March 10th, 2025 & 0.0 & First iteration of rough draft complete\\
\bottomrule
\end{tabularx}

~\newpage

\section{Symbols, Abbreviations and Acronyms}

\renewcommand{\arraystretch}{1.2}
\begin{tabular}{l l} 
  \toprule		
  \textbf{symbol} & \textbf{description}\\
  \midrule 
  T & Test\\
  SRS & Software Requirements Specification \\
  MOSS & Measure of Software Similarity \\
  UI & User Interface \\
  JSON & JavaScript Object Notation \\
  NPM & Node Package Manager \\
  VnV & Verification and Validation \\
  API & Application Programming Interface \\
  CI/CD & Continuous Integration/Continuous Deployment\\
  \bottomrule
\end{tabular}\\

\newpage

\tableofcontents

\listoftables %if appropriate

\listoffigures %if appropriate

\newpage

\pagenumbering{arabic}

\section{Functional Requirements Evaluation}
\subsubsection{Plagiarism Analysis Input Upload Tests}

Applies to all FRs involving uploading code snippets to the system. This refers
to test-FR-1 from the VnV Plan.

\begin{enumerate}

\item{\textbf{test-FR-1}}

Expected result: The script should run and successfully upload the files, without 
encountering any errors due to the upload functionality

Actual Result: No errors from uplading

Evaluation: Pass
					

\end{enumerate}

\subsubsection{Plagiarism Analysis Result Tests}

Applies to all FRs involving immediate output of plagiarism analysis. This
refer to test-FR-2 from the VnV Plan.

\begin{enumerate}

\item{\textbf{test-FR-2}}

Expected result: The analysis should not throw any errors while the script runs,
up until the analysis provides a response

Actual result: No errors were observed during script execution

Evaluation: Pass

					

\end{enumerate}

\subsubsection{Guide Documentation Generation Tests}

Applies to all FRs involving guide documentation generation for user. This
refers to test-FR-3 from the VnV plan.

\begin{enumerate}

\item{\textbf{test-FR-3}}

Expected result: Generation of documentation successful

Actual result: TBD when feature is implemented

Evaluation: TBD

\end{enumerate}

\subsubsection{Report Documentation Generation Tests}

Applies to all FRs involving analysis report documentation generation for 
user. This refers to test-FR-4 from the VnV plan.

\begin{enumerate}

\item{\textbf{test-FR-4}}	

Expected result: Generation of report is successful, with no errors encountered

Actual result: Report was successfully generated, no errors were found

Evaluation: Pass

\end{enumerate}

\subsubsection{Account Creation Tests}

Applies to all FRs involving creating an account within the system. This refers
to test-FR-5 and test-FR-6 from the VnV plan
\begin{enumerate}

\item{\textbf{test-FR-5}}

Expected result: An account should be successfully created, while the system avids any failure modes.

Actual result: Account successfully created, and no failure modes were exhibited

Evaluation: Pass

\item{\textbf{test-FR-6}}
		
Expected result: Account creation should fail and the system will await further input from the user

Actual result: Account creation fails, and the system does not proceed

Evaluation: Pass 

\end{enumerate}

\subsubsection{Account Login Tests}

Applies to all FRs involving logging into account within the system. This refers
to test-FR-7 and test-FR-8 in the VnV plan

\begin{enumerate}

\item{\textbf{test-FR-7}}

Expected result: Logging into system with correct existing credentials should
succeed, and no errors should arise from logging in

Actual result: Logging in with existing credentials succeeds with no errors

Evaluation: Pass

\item{\textbf{test-FR-8}}

Expected result: Logging in with invalid credentials should fail

Actual result: Logging in with invalid credentials fails

Evaluation: Pass
					
\end{enumerate}

\subsubsection{Result Visualization Tests}

Applies to all FRs involving visualizing plagiarism analysis results. This refers to test-FR-9 from the VnV plan.


\begin{enumerate}

\item{\textbf{test-FR-10}}
			
Expected result: A visualization is successfully created after uplading a .zip
file, with no errors observed during the process.

Actual result: A visualization was successfully created with no errors

Evaluation: Pass

\end{enumerate}



\section{Nonfunctional Requirements Evaluation}
% start here
\begin{longtable}[c]{|p{1.5cm}|p{1.5cm}|p{2.5cm}|p{2.5cm}|p{2.5cm}|p{1.5cm}|}
    \hline
    \textbf{ID} & \textbf{Reference Requirements} & \textbf{User Action} &
     \textbf{Expected Result} & \textbf{Actual Result} & \textbf{Result} \\
    \hline
    \endfirsthead
    \hline
    \textbf{ID} & \textbf{Reference Requirements} & \textbf{User Action} &
     \textbf{Expected Result} & \textbf{Actual Result} & \textbf{Result} \\
    \hline
    \endhead
    \hline
    test-LF-1 & LF-AR1, LF-AR2, LF-AR3, LF-SR1, LF-SR2, LF-SR3, LF-SR4 & 
    Developer reviews UI code and interface for consistency with the SRS 
    elements specified in VnV plan such as uniform colour palette. &
     UI should follow defined layout, typography, alignment, color palette, 
     tooltips, responsiveness, and font. & Developer verified uniform color 
     palette, tooltips, responsive design, and font consistency. All 
     requirements were met. & Pass \\
    \hline
    test-LF-2 & CR-SC3 & Developer utilizes accessibility tools to assess sufficient
    readability and alt-text for images & All images should have alt text, and text 
    should meet contrast and sizing standards. & Developer verified all images have 
    alt text and all text meets contrast and sizing requirements using Chrome DevTools. & Pass \\
    \hline
    test-UH-1 & UH-E1, UH-E3, UH-L1, UH-UP1, UH-UP2, OE-P1 & User goes through 
    the onboarding process and provides feedback on usability. & User finds 
    onboarding process intuitive with no major issues; usability rating should 
    be 4/5 or greater. & 5 users tested the onboarding process. All rated ease
     of use and clarity at 4/5 or greater. & Pass \\
    \hline
    test-UH-2 & UH-E2 & User navigates through key functions of code submission
     and report upload & User should complete each task with minimal clicks required. 
     & 5 users tested navigation to code submission and report upload, reporting 2-3 
     clicks each from upload to select to run. & Pass \\
    \hline
    test-UH-3 & UH-PI2 & Tester verifies that all pages and prompts are in 
    English (US) & No foreign languages should be present on any page or prompt. 
    & Tester verified that all pages and prompts were in English (US), no foreign
     languages found. & Pass \\
    \hline
    test-V-1 & MS-M1, OR-R2 & Developer reviews GitHub repository for historical 
    versions and changelogs & All previous versions should be stored with performance 
    reports and changelogs. & Developer confirmed that historical versions with 
    performance reports and changelogs are accessible on GitHub. & Pass \\
    \hline
    test-PS-1 & SR-P1 & Tester performs a data retention audit to identify any personally
     identifiable data & No personally identifiable data or user-generated data should be
      found in logs or filesystem. & Tester found no PII or user-generated data during 
      testing. Logs were clean. & Pass \\
    \hline
    test-PS-2 & SR-P2 & Tester verifies that backend endpoints use HTTPS encryption & All 
    endpoints should use HTTPS with proper security headers. & Tester confirmed all backend 
    endpoints used HTTPS with proper security headers as shown with REST API utilized. & Pass \\
    \hline
    test-PS-3 & CR-L1, CR-L2, CR-L3, CR-S2, CR-S5 & Legal audit of the system against 
    regulations like PIPEDA, FIPPA, etc. & System should adhere to legal standards 
    without any breaches. & Third-party tester verified the system's compliance with 
    PIPEDA, FIPPA, and other regulations. No breaches found. This is inherent by only 
    utilizing publicly available repos and data & Pass \\
    \hline
    test-DS-1 & MS-M3 & Tester inspects the GitHub repository for bug tracking. & 
    Bugs should be listed as issues in the repository. & Tester confirmed bugs are 
    listed in the 'issues' section on GitHub. Chores and fixes have been dealt with 
    as issues & Pass \\
    \hline
    test-DS-2 & MS-S1, CR-SC1 & Team inspects code to ensure adherence to software 
    development standards & Code should follow SOLID principles, and be documented 
    with appropriate tests. & Team inspected code, confirming adherence to SOLID 
    principles, documentation, and test presence. Code for backend application 
    follows direct interface hiding rules and currently complies. Our frontend 
    works on a component based development path and therefore adheres too. & Pass \\
    \hline
    test-D-1 & UH-L2, OE-P2 & Developer reviews user documentation for clarity, 
    accuracy, and helpfulness & Documentation should be clear, non-technical, 
    and helpful to users. & Help documentation is still to be developed per 
    extra but current README files exist to specify steps for application use 
    in clear and basic language for users. & Pass with exception to extra \\
    \hline
    test-PR-1 & PR-SL1, PR-C1 & System processes 500 code snippets for plagiarism
     analysis within 4 hours & Processing should complete in under 4 hours. & 
     System completed processing within 4 hours, performance met expectations. & Pass \\
    \hline
    test-PR-2 & PR-RFT1 & Tester submits a corrupted file and verifies system response 
    & System should reject the corrupted input with an error message without crashing. 
    & System rejected the corrupted input with an error message. Valid input processed 
    without issue.  i.e., non-python or tar package file creates error which frontend 
    appropriately communicates. & Pass \\
    \hline
    test-PR-3 & PR-PA1, PR-PA2 & Tester submits 500 code snippets for plagiarism analysis 
    and compares results with ground truth. & System should show over 90\% accuracy and 
    less than 5\% false positives. & Recently found method to test this through self-labelled datasets
    as it was not immediately possible from python800 datasets (no concept of similarity or ground truths)
    Have spent time labelling hundreds of code in manual fashion to assess this and have not had time to test it.  
    & Future Assessment \\
    \hline
    test-PR-4 & PR-SE1 & Developer adds a new feature, and all existing features are tested
     for regression. & All previous features should work as expected after the new feature 
     is added. & Regression tests confirmed that all features worked as expected after the 
     new feature was added but not server side yet & Fail\\
    \hline
    test-OE-1 & OE-IAS1 & User connects system to a cloud service (e.g., AWS or Azure) and 
    verifies successful connection. & System should successfully connect to cloud service 
    and display a confirmation message. & Cloud has yet to be connected. It is a later stage 
    development process & Future Assessment \\
    \hline
    test-OE-2 & OE-IAS2 & User deploys the system on a free hosting service
    and verifies system access after cloning from github. & System should be 
    accessible via a public URL and function as 
    expected. & System has not been deployed to AWS or other free hosting 
    services yet. Later stage development. 
    & Future Assessment \\
    test-OE-3 & OE-IAS3 & User authenticates via an external service (e.g., Google, GitHub) 
    and gains access to the UI. & System should authenticate user and redirect to 
    the home page. & User successfully authenticated via Google through Auth Zero 
    and was redirected to the home page. & Pass \\
    \hline
    test-M-1 & MS-M2 & Model release is accompanied by a report of metrics such as 
    accuracy, precision, etc. & A report should be generated with relevant metrics 
    for the model release. & Current model release has metrics on microsoft
    althought future releases will have a new report attached and that is part of later stage development & Future Assessment \\
    \hline
    test-M-2 & MS-S3 & User posts a request or issue on GitHub, verifying visibility
     to others & The request or issue should be visible to other users. & Request 
     posted on GitHub was visible to other users as expected. & Pass \\
    \hline
    test-M-3 & MS-A2 & Model code follows template, and components inherit 
    interfaces/classes. & Model components should follow the predefined template 
    and interface inheritance. & Code was inspected and confirmed that all 
    components inherited from interfaces/classes as required. Model app structure
    follows SOLID principles and model itself is masked in pytorch. & Pass \\
    \hline
    test-M-4 & MS-A1 & Training script receives two formats of training data 
    and completes an epoch. & Training script should successfully complete an 
    epoch with both data formats. & Training script completed successfully with 
    both data formats (java and python files) and have corresponding checkpoint
    files to show. & Pass \\
    \hline
    test-M-5 & MS-A3 & Each model layer/component is tested with an appropriate 
    input vector and verifies transformation. & Each layer/component should modify 
    the input vector according to its specific architecture. & Model layers were 
    tested, and all correctly transformed input vectors according to design. Pytorch 
    and scipy layers used are atomic in nature and can be mixed as desired & Pass \\
    \hline
    test-S-1 & SR-A1 & User enters valid credentials and accesses the UI. & User 
    should be authenticated and granted access to the UI. & User entered valid 
    credentials and successfully accessed the UI. & Pass \\
    \hline
    test-S-2 & SR-A1 & User submits code to API with valid auth token. & API 
    should allow code upload and return a success response. & API accepted 
    valid auth token and processed code upload successfully. & Pass \\
    \hline
    test-S-3 & SR-A1 & User attempts to access the API with invalid or missing 
    auth token. & API should deny access and return an unauthorized response. 
    & API denied access when provided with invalid or missing auth token. & Pass \\
    \hline
    \end{longtable}
	
\section{Comparison to Existing Implementation}	

\subsection{Overview of Existing Solutions}

\subsubsection{MOSS (Measure of Software Similarity)}
MOSS is currently the standard tool for detecting code plagiarism in academic settings. Developed at Stanford University, it works by:
\begin{itemize}
    \item Tokenizing code into a sequence of symbols
    \item Using document fingerprinting to detect similar code segments
    \item Generating a similarity score based on matching sequences
    \item Providing a web interface to view overlapping code segments
\end{itemize}

\subsubsection{Other Notable Solutions}
\begin{itemize}
    \item \textbf{JPlag}: Uses a token-based approach similar to MOSS, but with different tokenization strategies
    \item \textbf{PLAGGIE}: Java-specific plagiarism detection tool used in academic environments
    \item \textbf{CodeMatch}: Commercial solution that uses both tokenization and metric-based methods
\end{itemize}

\subsection{Our Implementation vs. Existing Solutions}

\subsubsection{Current Approach}
Our implementation leverages CodeBERT, a pre-trained model for programming language understanding, to perform code pair comparison. Key aspects include:
\begin{itemize}
    \item Utilization of transformer-based neural networks to understand code semantics
    \item Direct comparison of code pairs to determine similarity
    \item A modern, user-friendly interface for result visualization and analysis
\end{itemize}

\subsubsection{Functional Comparison}

\begin{table}[h]
\centering
\begin{tabular}{|p{3cm}|p{5cm}|p{5cm}|}
\hline
\textbf{Feature} & \textbf{MOSS} & \textbf{Our CodeBERT Implementation} \\
\hline
Detection Method & Token-based fingerprinting & Neural representation and semantic understanding \\
\hline
Language Support & Multiple languages & Multiple languages (supported by CodeBERT) \\
\hline
Semantic Understanding & Limited (syntax-focused) & Enhanced (captures semantic meaning) \\
\hline
Resistance to Obfuscation & Low-moderate & Potentially higher \\
\hline
Speed & Fast for large submissions & Currently slower for large-scale comparisons \\
\hline
Visualization & Basic web interface & Modern, interactive UI \\
\hline
\end{tabular}
\caption{Comparison between MOSS and our CodeBERT implementation}
\label{tab:comparison}
\end{table}

\subsection{Advantages of Our Approach}

\begin{enumerate}
    \item \textbf{Semantic Understanding}: Unlike MOSS's purely syntactic approach, CodeBERT can potentially understand the meaning behind code, making it more difficult to fool with non-functional additions.

    \item \textbf{Context Awareness}: Our implementation considers the context in which code appears, potentially reducing false positives from common solutions to standard problems.

    \item \textbf{Modern Interface}: Our user interface provides a more intuitive experience for instructors reviewing potential plagiarism cases.
\end{enumerate}

\subsection{Current Limitations and Trade-offs}

\begin{enumerate}
    \item \textbf{Computational Requirements}: Neural models like CodeBERT require more computational resources than traditional approaches like MOSS.

    \item \textbf{Development Maturity}: MOSS has been refined over decades, while our solution is still in the early stages of development.

    \item \textbf{Validation}: Our approach needs more extensive testing across different programming languages and assignments to prove its effectiveness.

    \item \textbf{Explainability}: Neural network decisions can be less transparent than token-matching approaches, making it potentially harder to explain why certain code pairs were flagged.
\end{enumerate}

\subsection{Design Decision Justification}

\begin{enumerate}
    \item \textbf{Choice of CodeBERT}: We selected CodeBERT over other models because it was specifically pre-trained on code from multiple programming languages, making it well-suited for understanding code semantics across different languages used in academic settings.

    \item \textbf{Focus on Pair Comparison}: While MOSS compares each submission against all others, our current approach focuses on direct pair comparison, allowing for more detailed analysis of potentially plagiarized code.

    \item \textbf{UI Priority}: We invested in a high-quality UI early in development to facilitate easier testing and validation of our detection results.
\end{enumerate}

\subsection{Future Improvements}

\begin{enumerate}
    \item \textbf{Hybrid Approach}: Combining neural understanding with traditional token-based methods to leverage strengths of both approaches.

    \item \textbf{Performance Optimization}: Improving the computational efficiency to handle large class submissions more effectively.

    \item \textbf{Tunable Sensitivity}: Implementing adjustable thresholds for different assignment types and programming languages.
\end{enumerate}

\section{Unit Testing}
\subsection{AnalyzeFiles Component}

\subsubsection{Renders the component}
\begin{itemize}
    \item \textbf{Description}: Ensures the \texttt{AnalyzeFiles} component renders correctly.
    \item \textbf{Type}: Automatic, Functional
    \item \textbf{Initial state}: None
    \item \textbf{Test case steps}:
    \begin{enumerate}
        \item Render the \texttt{AnalyzeFiles} component.
        \item Check for the presence of specific text.
    \end{enumerate}
    \item \textbf{Expected result}: The component renders with the text "Analyze a Dataset" and "Upload a dataset to analyze for potential code plagiarism. Use a ZIP file containing your project files."
\end{itemize}

\subsubsection{Displays an alert when no file is uploaded}
\begin{itemize}
    \item \textbf{Description}: Ensures an alert is displayed when no file is uploaded.
    \item \textbf{Type}: Automatic, Functional
    \item \textbf{Initial state}: None
    \item \textbf{Test case steps}:
    \begin{enumerate}
        \item Render the \texttt{AnalyzeFiles} component.
        \item Click the "Analyze Dataset" button.
    \end{enumerate}
    \item \textbf{Expected result}: An alert with the message "Please upload a dataset file first." is displayed.
\end{itemize}

\subsubsection{Displays an alert when no analysis name is entered}
\begin{itemize}
    \item \textbf{Description}: Ensures an alert is displayed when no analysis name is entered.
    \item \textbf{Type}: Automatic, Functional
    \item \textbf{Initial state}: None
    \item \textbf{Test case steps}:
    \begin{enumerate}
        \item Render the \texttt{AnalyzeFiles} component.
        \item Upload a valid ZIP file.
        \item Click the "Analyze Dataset" button.
    \end{enumerate}
    \item \textbf{Expected result}: An alert with the message "Please enter an analysis name." is displayed.
    
\end{itemize}

\subsubsection{Calls uploadFiles and displays success message when form is submitted}
\begin{itemize}
    \item \textbf{Description}: Ensures the \texttt{uploadFiles} function is called and a success message is displayed when the form is submitted.
    \item \textbf{Type}: Automatic, Functional
    \item \textbf{Initial state}: None
    \item \textbf{Test case steps}:
    \begin{enumerate}
        \item Render the \texttt{AnalyzeFiles} component.
        \item Upload a valid ZIP file.
        \item Enter an analysis name.
        \item Click the "Analyze Dataset" button.
    \end{enumerate}
    \item \textbf{Expected result}: The \texttt{uploadFiles} function is called with the file and analysis name, and a success message "Analysis started successfully" is displayed.
\end{itemize}

\subsubsection{Handles file drop event}
\begin{itemize}
    \item \textbf{Description}: Ensures the file drop event is handled correctly when a file is dropped on the component.
    \item \textbf{Type}: Automatic, Functional
    \item \textbf{Initial state}: None
    \item \textbf{Test case steps}:
    \begin{enumerate}
        \item Render the \texttt{AnalyzeFiles} component.
        \item Simulate a file drop event with a valid ZIP file.
    \end{enumerate}
    \item \textbf{Expected result}: The file is uploaded successfully.
\end{itemize}

\subsection{Index (Landing Page) Component}

\subsubsection{Renders the component}
\begin{itemize}
    \item \textbf{Description}: Ensures the \texttt{Index} component renders correctly.
    \item \textbf{Type}: Automatic, Functional
    \item \textbf{Initial state}: None
    \item \textbf{Test case steps}:
    \begin{enumerate}
        \item Render the \texttt{Index} component.
        \item Check for the presence of specific text.
    \end{enumerate}
    \item \textbf{Expected result}: The component renders with the text "Verify Academic and Contest Submissions" and "Empower educators and contest administrators with advanced AI-powered plagiarism detection. Ensure the integrity of student work and competition submissions."
\end{itemize}

\subsubsection{Displays loading spinner when loading}
\begin{itemize}
    \item \textbf{Description}: Ensures the loading spinner is displayed when the component is in the loading state.
    \item \textbf{Type}: Automatic, Functional
    \item \textbf{Initial state}: \texttt{isLoading} is true
    \item \textbf{Test case steps}:
    \begin{enumerate}
        \item Mock the \texttt{useAuth0} hook to return \texttt{isLoading} as true.
        \item Render the \texttt{Index} component.
    \end{enumerate}
    \item \textbf{Expected result}: The text "Preparing your experience..." is displayed.
\end{itemize}

\subsubsection{Calls loginWithRedirect when Get Started button is clicked}
\begin{itemize}
    \item \textbf{Description}: Ensures the \texttt{loginWithRedirect} function is called when the "Get Started" button is clicked.
    \item \textbf{Type}: Automatic, Functional
    \item \textbf{Initial state}: None
    \item \textbf{Test case steps}:
    \begin{enumerate}
        \item Render the \texttt{Index} component.
        \item Click the "Get Started" button.
    \end{enumerate}
    \item \textbf{Expected result}: The \texttt{loginWithRedirect} function is called.
\end{itemize}

\subsubsection{Navigates to /home when authenticated}
\begin{itemize}
    \item \textbf{Description}: Ensures the user is redirected to the \texttt{Home} page when authenticated.
    \item \textbf{Type}: Automatic, Functional
    \item \textbf{Initial state}: None
    \item \textbf{Test case steps}:
    \begin{enumerate}
        \item Mock the \texttt{useAuth0} hook to return \texttt{isAuthenticated} as true.
        \item Render the \texttt{Index} component.
    \end{enumerate}
    \item \textbf{Expected result}: The user is redirected to the \texttt{Home} page.
\end{itemize}

\subsubsection{Displays features section}
\begin{itemize}
    \item \textbf{Description}: Ensures the features section is displayed on the \texttt{Index} component.
    \item \textbf{Type}: Automatic, Functional
    \item \textbf{Initial state}: None
    \item \textbf{Test case steps}:
    \begin{enumerate}
        \item Render the \texttt{Index} component.
    \end{enumerate}
    \item \textbf{Expected result}: The features section is displayed with the text "Why Choose SyntaxSentinals?" along with the features.
\end{itemize}

\subsubsection{Displays how it works section}
\begin{itemize}
    \item \textbf{Description}: Ensures the "How It Works" section is displayed on the \texttt{Index} component.
    \item \textbf{Type}: Automatic, Functional
    \item \textbf{Initial state}: None
    \item \textbf{Test case steps}:
    \begin{enumerate}
        \item Render the \texttt{Index} component.
    \end{enumerate}
    \item \textbf{Expected result}: The "How It Works" section is displayed along with the steps.
\end{itemize}

\subsubsection{Displays footer}
\begin{itemize}
    \item \textbf{Description}: Ensures the footer is displayed on the \texttt{Index} component.
    \item \textbf{Type}: Automatic, Functional
    \item \textbf{Initial state}: None
    \item \textbf{Test case steps}:
    \begin{enumerate}
        \item Render the \texttt{Index} component.
    \end{enumerate}
    \item \textbf{Expected result}: The footer is displayed with the text "SyntaxSentinals" and the links.

\end{itemize}

\subsection{Upload File Box Component}
\subsubsection{Renders the component}
\begin{itemize}
    \item \textbf{Description}: Ensures the \texttt{UploadFileBox} component renders correctly.
    \item \textbf{Type}: Automatic, Functional
    \item \textbf{Initial state}: None
    \item \textbf{Test case steps}:
    \begin{enumerate}
        \item Render the \texttt{UploadFileBox} component.
    \end{enumerate}
    \item \textbf{Expected result}: The component renders with the text "Upload a ZIP file" and the file input.
\end{itemize}

\subsubsection{Uploads a valid file}
\begin{itemize}
    \item \textbf{Description}: Ensures a valid file is uploaded using the \texttt{UploadFileBox} component.
    \item \textbf{Type}: Automatic, Functional
    \item \textbf{Initial state}: None
    \item \textbf{Test case steps}:
    \begin{enumerate}
        \item Render the \texttt{UploadFileBox} component.
        \item Upload a valid ZIP file.
    \end{enumerate}
    \item \textbf{Expected result}: The file is uploaded successfully.
\end{itemize}

\subsubsection{Removes a file}
\begin{itemize}
    \item \textbf{Description}: Ensures a file is removed using the \texttt{UploadFileBox} component.
    \item \textbf{Type}: Automatic, Functional
    \item \textbf{Initial state}: A file is uploaded
    \item \textbf{Test case steps}:
    \begin{enumerate}
        \item Render the \texttt{UploadFileBox} component.
        \item Click the "Remove" button.
    \end{enumerate}
    \item \textbf{Expected result}: The file is removed.
\end{itemize}

\subsubsection{Clears the file list when clearFiles is true}
\begin{itemize}
    \item \textbf{Description}: Ensures the file list is cleared when the \texttt{clearFiles} prop is true.
    \item \textbf{Type}: Automatic, Functional
    \item \textbf{Initial state}: A file is uploaded
    \item \textbf{Test case steps}:
    \begin{enumerate}
        \item Render the \texttt{UploadFileBox} component with the \texttt{clearFiles} prop set to true.
    \end{enumerate}
    \item \textbf{Expected result}: The file list is cleared.
\end{itemize}

\subsection{NavBar Component}
\subsubsection{Renders the component}
\begin{itemize}
    \item \textbf{Description}: Ensures the \texttt{NavBar} component renders correctly.
    \item \textbf{Type}: Automatic, Functional
    \item \textbf{Initial state}: None
    \item \textbf{Test case steps}:
    \begin{enumerate}
        \item Render the \texttt{NavBar} component.
    \end{enumerate}
    \item \textbf{Expected result}: The component renders with the logo and navigation links.
    
\end{itemize}

\subsubsection{Displays the login button when not authenticated}
\begin{itemize}
    \item \textbf{Description}: Ensures the login button is displayed when the user is not authenticated.
    \item \textbf{Type}: Automatic, Functional
    \item \textbf{Initial state}: None
    \item \textbf{Test case steps}:
    \begin{enumerate}
        \item Mock the \texttt{useAuth0} hook to return \texttt{isAuthenticated} as false.
        \item Render the \texttt{NavBar} component.
    \end{enumerate}
    \item \textbf{Expected result}: The login button is displayed.
\end{itemize}

\subsubsection{Displays the logout button when authenticated}
\begin{itemize}
    \item \textbf{Description}: Ensures the logout button is displayed when the user is authenticated.
    \item \textbf{Type}: Automatic, Functional
    \item \textbf{Initial state}: None
    \item \textbf{Test case steps}:
    \begin{enumerate}
        \item Mock the \texttt{useAuth0} hook to return \texttt{isAuthenticated} as true.
        \item Render the \texttt{NavBar} component.
    \end{enumerate}
    \item \textbf{Expected result}: The logout button is displayed.
\end{itemize}

\subsubsection{Calls loginWithRedirect when Log In button is clicked}
\begin{itemize}
    \item \textbf{Description}: Ensures the \texttt{loginWithRedirect} function is called when the "Log In" button is clicked.
    \item \textbf{Type}: Automatic, Functional
    \item \textbf{Initial state}: None
    \item \textbf{Test case steps}:
    \item \begin{enumerate}
        \item Mock the \texttt{useAuth0} hook to return \texttt{isAuthenticated} as false.
        \item Render the \texttt{NavBar} component.
        \item Click the "Log In" button.
    \end{enumerate}
    \item \textbf{Expected result}: The \texttt{loginWithRedirect} function is called.
\end{itemize}

\section{Changes Due to Testing}

During the testing process, several issues were identified and addressed, leading to improvements in both functionality and reliability:

\begin{itemize}
\item \textbf{Upload File Handling}
\begin{itemize}
\item An issue was detected in \texttt{UploadBox.tsx}, where files were not properly validated before being uploaded. Additional validation logic was added to prevent incorrect file types from being processed.
\item A missing error message for unsupported file types was implemented to improve user feedback.
\end{itemize}
\item \textbf{Error Handling in File Processing}
\begin{itemize}
\item In \texttt{UploadResults.tsx}, tests revealed that errors during ZIP file processing were not handled gracefully. Additional error handling was introduced to catch and display meaningful error messages when invalid or corrupted ZIP files were uploaded.
\end{itemize}
\item \textbf{Form Validation Enhancements}
\begin{itemize}
\item The \texttt{AnalyzeFiles.tsx} component initially allowed form submission without an analysis name, causing backend errors. A required field check was added to prevent submissions without a name.
\end{itemize}
\item \textbf{Test Coverage Improvements}
\begin{itemize}
\item Some uncovered lines in \texttt{button.tsx} and \texttt{UploadBox.tsx} were identified through test coverage reports. Additional unit tests were written to cover these missing scenarios, increasing overall test coverage.
\item Server side code was not tested with a testing framework. This will be implemented in the future.
\end{itemize}
\item \textbf{Cloud}
\begin{itemize}
\item Cloud deployment is a future feature that will be implemented once the model is certified per priorities
\end{itemize}
\end{itemize}


\section{Automated Testing}
The tests mentioned in the \texttt{Unit Testing} section are automated using the Jest testing framework. The tests are run using the command \texttt{npm test}.

To ensure continuous integration and automated testing, we have set up a GitHub Actions workflow. The workflow is triggered on every push and pull request to the repository. It runs the tests on an Ubuntu environment with Node.js version 18.x.

\subsection{Testing Workflow Configuration}

The following is the configuration of the GitHub Actions workflow used for automated testing:

\begin{verbatim}
name: Run Frontend Tests

on: [push, pull_request]

jobs:
  test:
    runs-on: ubuntu-20.04

    strategy:
      matrix:
        node-version: [18.x]

    steps:
    - name: Checkout repository
      uses: actions/checkout@v4

    - name: Set up Node.js
      uses: actions/setup-node@v4
      with:
        node-version: ${{ matrix.node-version }}

    - name: Install dependencies
      run: npm install
      working-directory: ./src/frontend

    - name: Run tests
      run: npm test
      working-directory: ./src/frontend
\end{verbatim}

\subsection{Testing Workflow Steps}

\begin{itemize}
    \item \textbf{Checkout repository}: This step uses the \texttt{actions/checkout@v4} action to check out the repository code.
    \item \textbf{Set up Node.js}: This step uses the \texttt{actions/setup-node@v4} action to set up Node.js with the specified version (18.x).
    \item \textbf{Install dependencies}: This step runs \texttt{npm install} to install the necessary dependencies for the frontend.
    \item \textbf{Run tests}: This step runs \texttt{npm test} to execute the Jest tests for the frontend.
\end{itemize}

By using this automated testing workflow, we ensure that our code is continuously tested and validated, providing immediate feedback on any issues that may arise from changes in the codebase.

\subsection{Test Coverage Workflow Configuration}

In addition to running unit tests, we have configured a separate GitHub Actions workflow to generate a test coverage report using Jest. This helps us monitor the extent to which our code is covered by automated tests.

The following is the configuration for the test coverage workflow:

\begin{verbatim}
name: Run Frontend Unit Tests

on: [push, pull_request]

jobs:
coverage:
runs-on: ubuntu-20.04

steps:
  - name: Checkout repository
    uses: actions/checkout@v3

  - name: Run Jest Coverage Report
    uses: ArtiomTr/jest-coverage-report-action@v2
    with:
      working-directory: ./src/frontend
      github-token: ${{ secrets.GITHUB_TOKEN }}

\end{verbatim}

\subsection{Test Coverage Workflow Steps}

\begin{itemize}
\item \textbf{Checkout repository}: This step uses the \texttt{actions/checkout@v3} action to check out the repository code.
\item \textbf{Run Jest Coverage Report}: This step uses the \\ \texttt{ArtiomTr/jest-coverage-report-action@v2} action to generate a coverage report for the Jest tests in the frontend directory.
\end{itemize}

By incorporating test coverage reporting into our workflow, we can ensure that our automated tests provide sufficient coverage of the codebase, identifying untested areas and improving software reliability.
		
\section{Trace to Requirements}

\begin{center}
    \begin{longtable}{|>{\centering\arraybackslash}p{4cm}|>{\centering\arraybackslash}p{10cm}|}
    \hline
    \textbf{Test ID} & \textbf{Requirements} \\
    \hline
    \endfirsthead
    
    \hline
    \textbf{Test ID} & \textbf{Requirements} \\
    \hline
    \endhead
    
    test-FR-1 & FR-1 \\
    \hline
    test-FR-2 & FR-2, FR-4, FR-5 \\
    \hline
    test-FR-3 & FR-3 \\
    \hline
    test-FR-4 & FR-6 \\
    \hline
    test-FR-5 & FR-7 \\
    \hline
    test-FR-6 & FR-7 \\
    \hline
    test-FR-7 & FR-8 \\
    \hline
    test-FR-8 & FR-8 \\
    \hline
    test-FR-9 & FR-9 \\
    \hline
    test-LF-1 & LF-AR1, LF-AR2, LF-AR3, LF-SR1, LF-SR2, LF-SR3, LF-SR4 \\
    \hline
    test-LF-2 & CR-SC3 \\
    \hline
    test-UH-1 & UH-E1, UH-E3, UH-L1, UH-UP1, UH-UP2, OE-P1 \\
    \hline
    test-UH-2 & UH-E2 \\
    \hline
    test-UH-3 & UH-PI2 \\
    \hline
    test-V-1 & MS-M1, OR-R2 \\
    \hline
    test-PS-1 & SR-P1 \\
    \hline
    test-PS-2 & SR-P2 \\
    \hline
    test-PS-3 & CR-L1, CR-L2, CR-L3, CR-S2, CR-S5 \\
    \hline
    test-DS-1 & MS-M3 \\
    \hline
    test-DS-2 & MS-S1, CR-SC1 \\
    \hline
    test-D-1 & UH-L2, OE-P2 \\
    \hline
    test-PR-1 & PR-SL1, PR-C1 \\
    \hline
    test-PR-2 & PR-PA1, PR-PA2 \\
    \hline
    test-PR-3 & PR-SE1 \\
    \hline
    test-OE-1 & OE-IAS1 \\
    \hline
    test-OE-2 & OE-IAS2 \\
    \hline
    test-OE-3 & OE-IAS3 \\
    \hline
    test-M-1 & MS-M2 \\
    \hline
    test-M-2 & MS-S3 \\
    \hline
    test-M-3 & MS-A2 \\
    \hline
    test-M-4 & MS-A1 \\
    \hline
    test-M-5 & MS-A3 \\
    \hline
    test-S-1 & SR-A1 \\
    \hline
    test-S-2 & SR-A1 \\
    \hline
    test-S-3 & SR-A1 \\
    \hline
  
    \caption{Traceability Matrix}
    \end{longtable}
    \end{center}
		
\section{Trace to Modules}	
Note: M7 encapsulates all its submodules.
\begin{center}
    \begin{longtable}{|>{\centering\arraybackslash}p{4cm}|>{\centering\arraybackslash}p{10cm}|}
    \hline
    \textbf{Test ID} & \textbf{Requirements} \\
    \hline
    \endfirsthead
    
    \hline
    \textbf{Test ID} & \textbf{Module ID} \\
    \hline
    \endhead
    
    test-FR-1 &  M2\\
    \hline
    test-FR-2 & M4, M5, M6, M7, M8, M9\\
    \hline
    test-FR-3 & M2 \\
    \hline
    test-FR-4 & M6, M9 \\
    \hline
    test-FR-5 & M1 \\
    \hline
    test-FR-6 & M1 \\
    \hline
    test-FR-7 & M1 \\
    \hline
    test-FR-8 & M1 \\
    \hline
    test-FR-9 & M3 \\
    \hline
    test-LF-1 & All User Interface Modules \\
    \hline
    test-LF-2 & All User Interface Modules \\
    \hline
    test-UH-1 & All User Interface Modules \\
    \hline
    test-UH-2 & M2, M6 \\
    \hline
    test-UH-3 & M2, M6 \\
    \hline
    test-V-1 &  M7, M8 \\
    \hline
    test-PS-1 &  All Modules\\
    \hline
    test-PS-2 &  M6, M9 \\
    \hline
    test-PS-3 & All Modules \\
    \hline
    test-DS-1 &  Module Independent\\
    \hline
    test-DS-2 & All Modules\\
    \hline
    test-D-1 & M2\\
    \hline
    test-PR-1 &  M7, M8\\
    \hline
    test-PR-4 &  M2, M7, M9\\
    \hline
    test-PR-5 &  M7, M8\\
    \hline
    test-PR-6 &  Module Independent\\
    \hline
    test-OE-1 & Module Independent\\
    \hline
    test-OE-2 & Module Independent\\
    \hline
    test-OE-3 & Module Independent\\
    \hline
    test-M-1 & M7, M8 \\
    \hline
    test-M-2 & Module Independent\\
    \hline
    test-M-3 & M7 \\
    \hline
    test-M-4 & M7 \\
    \hline
    test-M-5 & M7 \\
    \hline
    test-S-1 & M1 \\
    \hline
    test-S-2 & M1 \\
    \hline
    test-S-3 & M1 \\
    \hline
  
    \caption{Traceability Matrix}
    \end{longtable}
    \end{center}

\section{Code Coverage Metrics}

The below report is for the frontend codebase of SyntaxSentinals. The backend has not been tested with a testing framework yet.
Instead we ensure the backend is working as expected by manually testing the API endpoints using Postman and ensuring the server is sent the 
expected request via the frontend testing suite. \\

The test coverage report provides the following key metrics:

\begin{itemize}
\item \textbf{Statements}: 94.35\% (234/248)
\item \textbf{Branches}: 81.03\% (47/58)
\item \textbf{Functions}: 86.79\% (46/53)
\item \textbf{Lines}: 96.18\% (227/236)
\end{itemize}

\newpage

Below is a breakdown of coverage by individual files:

\begin{table}[h]
    \centering
    \begin{tabular}{|l|c|c|c|c|}
    \hline
    \textbf{Directory} & \textbf{Statements} & \textbf{Branches} & \textbf{Functions} & \textbf{Lines} \\
    \hline
    Features.tsx & 100\% & 100\% & 100\% & 100\% \\
    Navbar.tsx & 100\% & 100\% & 100\% & 100\% \\
    UploadBox.tsx & 96.42\% & 84.21\% & 93.33\% & 98.14\% \\
    GetStarted.tsx & 100\% & 100\% & 100\% & 100\% \\
    Home.tsx & 90.32\% & 69.56\% & 73.91\% & 93.16\% \\
    Button.tsx & 100\% & 66.66\% & 100\% & 100\% \\
    utils.ts & 100\% & 100\% & 100\% & 100\% \\
    AnalyzeFiles.tsx & 91.52\% & 75\% & 75\% & 94.54\% \\
    UploadResults.tsx & 89.23\% & 63.63\% & 72.72\% & 91.93\% \\
    \hline
    \end{tabular}
    \caption{Test Coverage Breakdown by Directory}
    \label{tab:coverage}
\end{table}

Some uncovered lines include:

\begin{itemize}
\item \textbf{UploadBox.tsx}: Line 26.
\item \textbf{button.tsx}: Line 44.
\item \textbf{AnalyzeFiles.tsx}: Lines 22, 25, 76.
\item \textbf{UploadResults.tsx}: Lines 20, 23, 50-51, 89.
\end{itemize}

These uncovered areas indicate potential gaps in test cases that should be addressed to achieve full coverage.

\bibliographystyle{plainnat}
\bibliography{../../refs/References}

\newpage{}
\section*{Appendix --- Reflection}

The information in this section will be used to evaluate the team members on the
graduate attribute of Reflection.

The purpose of reflection questions is to give you a chance to assess your own
learning and that of your group as a whole, and to find ways to improve in the
future. Reflection is an important part of the learning process.  Reflection is
also an essential component of a successful software development process.  

Reflections are most interesting and useful when they're honest, even if the
stories they tell are imperfect. You will be marked based on your depth of
thought and analysis, and not based on the content of the reflections
themselves. Thus, for full marks we encourage you to answer openly and honestly
and to avoid simply writing ``what you think the evaluator wants to hear.''

Please answer the following questions.  Some questions can be answered on the
team level, but where appropriate, each team member should write their own
response:


\begin{enumerate}
  \item What went well while writing this deliverable?
  
  It was relatively straightforward to map out the tests we needed to focus on 
  based on the Verification and Validation (VnV) plan document. The VnV plan 
  provided clear expectations for the necessary tests, making it easier to 
  structure the deliverable around those objectives. We didn't need to perform 
  as much high-level reassessment of the overall design as we initially thought 
  we might and had to for previous deliverables. This made the process feel 
  more smooth, as we could concentrate on detailing how the tests would be 
  executed and aligning them with the deliverables outlined earlier in the 
  project. The clear documentation we had so far played a significant role in 
  creating a sense of continuity, which made writing this deliverable easier.

  \item What pain points did you experience during this deliverable, and how
    did you resolve them?

    One of the main challenges we encountered was coming up with compelling 
    data to verify the non-functional requirements for the project. While we 
    were able to conduct usability tests with peers who rated our UI on a scale 
    from 1 to 5, we felt unsure about whether this feedback would provide 
    enough substance to support the NFRs. We sought advice from our TA, who reassured 
    us that such usability tests, particularly when supplemented by qualitative 
    feedback and user comments, could be deemed a reasonable and valid dataset 
    for justifying the passing of NFR tests. This gave us the confidence to proceed 
    with the existing data and treat it as adequate for demonstrating that the UI 
    was user-friendly and met our usability NFRs.

    Another pain point revolved around selecting the right framework for frontend 
    testing. There was some ambiguity about which testing framework 
    would best suit the needs of our application, especially given the variety 
    of tools and options available for testing JavaScript-based web applications. 
    We resolved this by conducting thorough research into various options. This let 
    us determine that Jest would be the most appropriate choice for our project. 
    Jest's'ease of use, and the fact that it was already well integrated with React 
    (the framework we were using for the frontend) made it a great fit.

  \item Which parts of this document stemmed from speaking to your client(s) or
  a proxy (e.g. your peers)? Which ones were not, and why?

  The majority of this document stemmed directly from team discussions. 
  Since we were working closely together and had a good grasp of the 
  technical specifications, most of the deliverable's content was informed 
  by our team's perspectives and experience. The only sections which required 
  client/peer assessment was anything related to the usability tests. As developers, 
  it was difficult for us to maintain impartiality about the UI's intuitiveness and 
  ease of use. We needed unbiased feedback to assess how intuitive and usable our 
  interface truly was, so we conducted usability tests with external users 
  (mainly classmates) to gather their insights. Their feedback on how easily they 
  could interact with the interface was necessary to assign a pass or fail for some tests. 
  Thus, while most of the document was internally driven, the usability testing portion 
  was based on direct feedback from external stakeholders to ensure that our assumptions 
  about the UI held up in real-world use.

  \item In what ways was the Verification and Validation (VnV) Plan different
  from the activities that were actually conducted for VnV?  If there were
  differences, what changes required the modification in the plan?  Why did
  these changes occur?  Would you be able to anticipate these changes in future
  projects?  If there weren't any differences, how was your team able to clearly
  predict a feasible amount of effort and the right tasks needed to build the
  evidence that demonstrates the required quality?  (It is expected that most
  teams will have had to deviate from their original VnV Plan.)

While the VnV plan provided a great starting point, we did encounter some 
deviations from the original plan as the project evolved. Aspects such as cloud 
deployment and metrics for the NLP model ended up being much more complex and 
time-consuming than we originally anticipated. We had not factored in the 
amount of work required for cloud integration and the creation of monitoring 
tools for the NLP model's performance, nor how much work it would be to get 
the project to a stage where it was ready for either of those components. These changes 
meant that certain tests had to be postponed to a later phase of the project, and we had to 
adjust the scope of the tests that could be conducted in the current phase.

Another area of change was related to user notification systems. Initially, the VnV 
plan assumed that notifications about long processing times (e.g., 10 minutes or more) 
would be shown directly in the UI. However, after some reflection and considering user 
behavior, we realized that it was unlikely that users would remain engaged with the UI 
during long processing times. Therefore, we decided to send these notifications via 
email only instead, which led to a change in the VnV testing scope. These modifications were 
a result of our growing understanding of user expectations. We found that such changes 
can be anticipated to some extent in future projects if we invest more time in the 
early stages to gather feedback, conduct prototyping, and set clear technical expectations.

While it was difficult to predict all of these changes upfront, the team was able to adjust 
quickly thanks to open communication and ascertaining of direction. If we had spent more 
time in earlier phases fully scoping out all potential technical complexities and user 
scenarios, we might have been able to avoid some of these surprises. That is a goal for 
future projects we participate in.
\end{enumerate}

\end{document}

\documentclass{article}

\usepackage{tabularx}
\usepackage{booktabs}

\title{Problem Statement and Goals\\\progname}

\author{\authname}

\date{}

%% Comments

\usepackage{color}

\newif\ifcomments\commentstrue %displays comments
%\newif\ifcomments\commentsfalse %so that comments do not display

\ifcomments
\newcommand{\authornote}[3]{\textcolor{#1}{[#3 ---#2]}}
\newcommand{\todo}[1]{\textcolor{red}{[TODO: #1]}}
\else
\newcommand{\authornote}[3]{}
\newcommand{\todo}[1]{}
\fi

\newcommand{\wss}[1]{\authornote{blue}{SS}{#1}} 
\newcommand{\plt}[1]{\authornote{magenta}{TPLT}{#1}} %For explanation of the template
\newcommand{\an}[1]{\authornote{cyan}{Author}{#1}}

%% Common Parts

\newcommand{\progname}{Software Engineering} % PUT YOUR PROGRAM NAME HERE
\newcommand{\authname}{Team 2, SyntaxSentinals
\\ Lucas Chen
\\ Dennis Fong
\\ Mohammad Mohsin Khan
\\ Julian Cecchini
\\ Luigi Quattrociocchi} % AUTHOR NAMES                  

\usepackage{hyperref}
    \hypersetup{colorlinks=true, linkcolor=blue, citecolor=blue, filecolor=blue,
                urlcolor=blue, unicode=false}
    \urlstyle{same}
                                


\begin{document}

\maketitle

\begin{table}[hp]
\caption{Revision History} \label{TblRevisionHistory}
\begin{tabularx}{\textwidth}{llX}
\toprule
\textbf{Date} & \textbf{Developer(s)} & \textbf{Change}\\
\midrule
Sept 23 & Dennis Fong & Add problem statement\\
Sept 24 & Julian Cecchini & Revise problem statement and add goals\\
\bottomrule
\end{tabularx}
\end{table}

\section{Problem Statement}

Plagiarism detection in code today is not sufficient. There are too many
workarounds to be able to avoid being flagged. With the new advances in natural
language and programming language understanding, none of it has been adapted to
plagiarism detection. Proper plagiarism detection helps uphold the merit of
one's achievements, and the addition of semantic understanding of code can help
navigate the thin border between plagiarism or the lack thereof. 
% \wss{You should check your problem statement with the
% \href{https://github.com/smiths/capTemplate/blob/main/docs/Checklists/ProbState-Checklist.pdf}
% {problem statement checklist}.} 

% \wss{You can change the section headings, as long as you include the required
% information.}

\subsection{Problem}
The Measure of Software Similarity algorithm, or Moss algorithm for short, is
the current standard for plagiarism detection of code. Broadly speaking, this
algorithm works by comparing tokenized code snippets and assigning a similarity
score without any weighting based on the complexity of the line being examined.
In otherwords, there is an inherent lack of semantic understanding for the code
being examined. This gives rise to a major flaw in the Moss algorithm, which is
that benign lines of code can be added to a program that do not improve or
change functionality but still serve to create an illusion of difference in the
eyes of the algorithm. Therefore, even with the Moss algorithm in play, students
can easily plagiarize the work of others. Ideally, students should get by on the
merit of their own work alone, and a better plagiarism detection can help
realize this.

\subsection{Inputs and Outputs}
Input: The problem will take two or more snippets of code for comparison (n>=2
code snippets). \\
Output: The desired output is a similarity score between every pairing of the
code snippets, and will provide a threshold score to decide whether each score
indicates plagiarism or not. May also provide an overall score to indicate if
plagariasm is suspected somewhere in the dataset provided. (n choose 2 scores, 1
threshold, and 1 overall score for (n C 2) +2 scores). 

% \wss{Characterize the problem in terms of ``high level'' inputs and outputs.  
% Use abstraction so that you can avoid details.}

\subsection{Stakeholders}
The primary stakeholders in this project are professors in any computing and
software department, and students enrolled in courses where coding is prevalent.
Professors have been identified as stakeholders since they are the people who
will be looking out for plagiarism within their own courses. This project
provides a tool to give professors the ability to make better predictions on
plagiarism. Another stakeholder would be students for two reasons. It would be
key to correctly identify the hardwork of a student to prevent others from
stealing credit from them, and it would also be critical that a student does not
have their hardwork misidentified as another's as it would unjustly punish the
original creator. Therefore, the project team must have in mind that we minimize
the chance that an innocent student is punished, and maximize the chance that
students have their hardwork correctly attributed to themselves alone. Lastly,
an additional stakeholder could be administrative bodies of schools who would
care to incorporate/regulate the use of this detector in their faculty, or give
lessens/awareness about it within an official capacity (i.e., meetings or
training sessions)

\subsection{Environment}
This solution will operate on a device, where two files will be fed to a model.
The model will leverage hardware provided on the cloud.


% \wss{Hardware and software environment}

\section{Goals}

\begin{center}
\hspace*{-1cm}
\begin{tabular}{ | p{3cm} | p{6cm} | p{6cm} | }
\hline
Goal & Explanation & Reason \\
\hline
Ease of Use & Detector has an intuitive way to insert data and obtain results &
This application is expected to be used as a secondary tool for
teachers/professors when administering assignments. It should not require
in-depth learning, or it will be too inconvenient as an assistant tool for
detecting plagiarism. (Measured by actions to complete analysis)\\
\hline
Clarity of Output & Detector explains how to interpret outputs clearly, leaving
no ambiguity in whether plagiarism is suspected & If the user does not
comprehend the output, it may result in unjust accusations or undetected
plagiarism. (Measured by lines of description or number of users who correctly
interpret output)\\
\hline
Real-Time Processing & The detector computes results on a dataset of code
snippets quickly, enabling professors to incorporate them into evaluations &
Since multiple assignments are administered over several weeks, the detector
must be fast enough to be realistic for daily use. (Measured by execution
time)\\
\hline
Ethical Accuracy & The detector prioritizes minimizing false positives over
false negatives & In this case, a false positive could cause harm to an innocent
student, while a false negative allows a violation to go unnoticed. The focus is
on protecting innocent students. (Measured by false positives and negatives
using recall, precision, etc.)\\
\hline
\end{tabular}
\end{center}


 
\section{Stretch Goals}
\begin{itemize}
  \item Plagiarism detector is proven to outperform Moss across several test
  sets.
  \item Different LLM architectures will be benchmarked against Moss to gauge
  most optimal archiecture.
  \item Enhance Moss with our findings to improve on the base algorithm (also
  necessary if no clear progress can be made in direction of training LLM)
  
\end{itemize}

\section{Challenge Level and Extras}
This project has been assigned a difficulty level of general, and may be subject
to change. The aim is to use well known techniques that have been extensively
researched, which may push the difficulty to an advanced level, depending on the
complexity and feasibility of the research. 


The team intends to build an interface to support the project along with a user
manual that provides information on how to utilize the interface, for a total of
two extras. More ideas for extras can be added in the future.

% \wss{State your expected challenge level (advanced, general or basic).  The
% challenge can come through the required domain knowledge, the implementation
% or something else.  Usually the greater the novelty of a project the greater
% its challenge level.  You should include your rationale for the selected
% level. Approval of the level will be part of the discussion with the
% instructor for approving the project.  The challenge level, with the approval
% (or request) of the instructor, can be modified over the course of the term.}

% \wss{Teams may wish to include extras as either potential bonus grades, or to
% make up for a less advanced challenge level.  Potential extras include
% usability testing, code walkthroughs, user documentation, formal proof,
% GenderMag personas, Design Thinking, etc.  Normally the maximum number of
% extras will be two.  Approval of the extras will be part of the discussion
% with the instructor for approving the project.  The extras, with the approval
% (or request) of the instructor, can be modified over the course of the term.}

\newpage{}

\section*{Appendix --- Reflection}

% \wss{Not required for CAS 741}

The purpose of reflection questions is to give you a chance to assess your own
learning and that of your group as a whole, and to find ways to improve in the
future. Reflection is an important part of the learning process.  Reflection is
also an essential component of a successful software development process.  

Reflections are most interesting and useful when they're honest, even if the
stories they tell are imperfect. You will be marked based on your depth of
thought and analysis, and not based on the content of the reflections
themselves. Thus, for full marks we encourage you to answer openly and honestly
and to avoid simply writing ``what you think the evaluator wants to hear.''

Please answer the following questions.  Some questions can be answered on the
team level, but where appropriate, each team member should write their own
response:


\begin{enumerate}
    \item What went well while writing this deliverable? 
    
    During this deliverable, everything mostly went smoothly. Members were
    assigned different parts, and after everyone had a solid understanding of
    what our project is, completing our assigned sections had little to no
    conflict.
    
    \item What pain points did you experience during this deliverable, and how
    did you resolve them?

    The main paint points came from gaining a grasp of the project and its
    scope. The project itself is not simple by any standards. Figuring out what
    we are doing and ensuring everyone on the team had the same understanding
    was a challenge, which was resolved through a discussion. Aside from that,
    there were no major paint points.

    
    \item How did you and your team adjust the scope of your goals to ensure
    they are suitable for a Capstone project (not overly ambitious but also of
    appropriate complexity for a senior design project)?

    At times, the project felt overly ambitious. However, after thorough review
    and research, the feasibility and difficulty of the project seemed much more
    manageable, and not overly ambitious. The goals we have placed are
    reasonable, measurable, and are all things that can be achieved and improved
    on as the project goes forward.
\end{enumerate}  

\end{document}
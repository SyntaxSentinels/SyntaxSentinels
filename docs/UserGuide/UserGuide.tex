\documentclass{article}

\usepackage{booktabs}
\usepackage{tabularx}
\usepackage{hyperref}
\usepackage{listings}
\usepackage{fancyvrb}
\usepackage{geometry}
\geometry{margin=1in}

\input{../Comments}
\input{../Common}

\title{User Guide\\\progname}
\author{\authname}
\date{}

\begin{document}

\maketitle
\thispagestyle{empty}

\begin{table}[h]
\caption{Revision History} \label{TblRevisionHistory}
\begin{tabularx}{\textwidth}{llX}
\toprule
\textbf{Date} & \textbf{Developer(s)} & \textbf{Change}\\
\midrule
4/4/2025 & Mohammad Mohsin Khan and Lucas Chen & User Guide\\
\bottomrule
\end{tabularx}
\end{table}

\newpage

\section*{SyntaxSentinels User Guide}
This guide covers the setup and management of the \textbf{SyntaxSentinels} system, including the Compute Server, Frontend, and Express Server. It includes installation instructions, environment configuration, common tasks, and debugging tips.

\tableofcontents
\newpage

\section{Overview}
The SyntaxSentinels system is divided into three major components:

\begin{itemize}
    \item \textbf{Compute Server}: Processes background jobs using Python and interacts with AWS (S3, SQS).
    \item \textbf{Frontend}: A client-facing application set up with Node.js.
    \item \textbf{Express Server}: Handles API requests and integrates with AWS and Firebase.
\end{itemize}

Each component has its own setup process and environment variables. This guide provides a step-by-step walkthrough for installation, configuration, and debugging common issues.

\section{System Components}

\subsection{Compute Server}
\begin{itemize}
    \item \textbf{Language}: Python 3.11+
    \item \textbf{Key tasks}:
    \begin{itemize}
        \item Virtual environment creation
        \item Dependency installation using \texttt{requirements.txt}
        \item Running the worker process to handle background jobs from an SQS queue
    \end{itemize}
\end{itemize}

\subsection{Frontend}
\begin{itemize}
    \item \textbf{Language}: JavaScript (Node.js)
    \item \textbf{Key tasks}:
    \begin{itemize}
        \item Installing Node.js packages via \texttt{npm install}
        \item Configuring environment variables for Auth0 authentication and API integration
    \end{itemize}
\end{itemize}

\subsection{Express Server}
\begin{itemize}
    \item \textbf{Language}: JavaScript (Node.js)
    \item \textbf{Key tasks}:
    \begin{itemize}
        \item Installing Node.js packages via \texttt{npm install}
        \item Configuring environment variables for authentication (Auth0), AWS services, and Firebase
        \item Running the Express server for API endpoints
    \end{itemize}
\end{itemize}

\section{Prerequisites}
\begin{itemize}
    \item \textbf{Python 3.11+} (verify with \texttt{python --version})
    \item \textbf{Node.js and npm} (verify with \texttt{node --version} and \texttt{npm --version})
    \item A compatible shell:
    \begin{itemize}
        \item Windows: Command Prompt or PowerShell
        \item Linux/macOS: Standard terminal
    \end{itemize}
\end{itemize}

\section{Setup Instructions}

\subsection{Virtual Environment and Dependencies}

\subsubsection*{Compute Server}

\begin{Verbatim}[fontsize=\small]
cd backend
python -m venv .venv
\end{Verbatim}

Activate the environment:

\textbf{Windows Command Prompt:}
\begin{Verbatim}[fontsize=\small]
.venv\Scripts\activate.bat
\end{Verbatim}

\textbf{Windows PowerShell:}
\begin{Verbatim}[fontsize=\small]
.venv\Scripts\Activate.ps1
\end{Verbatim}

\textbf{Linux/macOS:}
\begin{Verbatim}[fontsize=\small]
source .venv/bin/activate
\end{Verbatim}

Install dependencies:
\begin{Verbatim}[fontsize=\small]
pip install -r requirements.txt
\end{Verbatim}

\subsubsection*{Frontend}

\begin{Verbatim}[fontsize=\small]
cd frontend
npm install
\end{Verbatim}

\subsubsection*{Express Server}

\begin{Verbatim}[fontsize=\small]
cd server
npm install
\end{Verbatim}

\subsection{Environment Variables}

Create a \texttt{.env} file for each component as described below.

\subsubsection*{Compute Server}

\begin{tabularx}{\textwidth}{lXl}
\toprule
\textbf{Variable} & \textbf{Description} & \textbf{Example} \\
\midrule
AWS\_REGION & AWS region location & \texttt{us-east-1} \\
AWS\_ACCESS\_KEY\_ID & AWS access key ID & \texttt{(none)} \\
AWS\_SECRET\_ACCESS\_KEY & AWS secret access key & \texttt{(none)} \\
S3\_BUCKET\_NAME & S3 bucket name & \texttt{syntax-sentinels-uploads} \\
SQS\_QUEUE\_URL & SQS job queue URL & \texttt{https://sqs...} \\
EXPRESS\_API\_URL & Express server URL & \texttt{http://localhost:3000/api} \\
\bottomrule
\end{tabularx}

\subsubsection*{Frontend}

\begin{tabularx}{\textwidth}{lXl}
\toprule
\textbf{Variable} & \textbf{Description} & \textbf{Example} \\
\midrule
VITE\_AUTH0\_DOMAIN & Auth0 domain & \texttt{myauth0domain.us.auth0.com} \\
VITE\_AUTH0\_CLIENT\_ID & Auth0 client ID & \texttt{123EXAMPLE} \\
VITE\_AUTH0\_AUDIENCE & Auth0 audience & \texttt{https://myauth0domain...} \\
VITE\_API\_URL & Express API URL & \texttt{http://localhost:3001/api} \\
\bottomrule
\end{tabularx}

\subsubsection*{Express Server}

(You can format the long environment variables in tabularx just like above, or break into multiple tables if needed.)

\section{Running the Servers}

\subsection{Compute Server}
\begin{enumerate}
    \item Activate the virtual environment.
    \item Set all environment variables.
    \item Run the worker process:
\begin{Verbatim}[fontsize=\small]
python worker.py
\end{Verbatim}
\end{enumerate}

\subsection{Frontend}
\begin{Verbatim}[fontsize=\small]
npm run dev
\end{Verbatim}

\subsection{Express Server}
\begin{Verbatim}[fontsize=\small]
npm start
\end{Verbatim}

\section{Debugging and Troubleshooting}

\subsection*{1. Virtual Environment Issues}
\begin{itemize}
    \item Make sure it's activated. If not:
\begin{Verbatim}[fontsize=\small]
source .venv/bin/activate
\end{Verbatim}
    \item Reinstall dependencies:
\begin{Verbatim}[fontsize=\small]
pip install --force-reinstall -r requirements.txt
\end{Verbatim}
\end{itemize}

\subsection*{2. Node.js Dependency Errors}
\begin{Verbatim}[fontsize=\small]
npm install
rm -rf node_modules
npm cache clean --force
npm install
\end{Verbatim}

\subsection*{3. Environment Variable Issues}
Double-check your \texttt{.env} files. Ensure sensitive values are correctly quoted and loaded.

\subsection*{4. Port Conflicts}
Check which processes are using ports \texttt{3000}, \texttt{3001} and kill or reconfigure as necessary.

\subsection*{5. AWS/Firebase Issues}
Use:
\begin{Verbatim}[fontsize=\small]
aws sts get-caller-identity
\end{Verbatim}

\section{Additional Tips}
\begin{itemize}
    \item Keep documentation updated.
    \item Use Git for configuration tracking.
    \item Backup sensitive files securely.
    \item Test components in isolation.
    \item Mirror environments between dev and prod.
\end{itemize}

\end{document}

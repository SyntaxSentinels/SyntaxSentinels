\documentclass{article}

\usepackage{tabularx}
\usepackage{booktabs}

\title{Reflection and Traceability Report on \progname}

\author{\authname}

\date{}

\input{../Comments}
\input{../Common}

\begin{document}

\maketitle

\plt{Reflection is an important component of getting the full benefits from a
learning experience.  Besides the intrinsic benefits of reflection, this
document will be used to help the TAs grade how well your team responded to
feedback.  Therefore, traceability between Revision 0 and Revision 1 is and
important part of the reflection exercise.  In addition, several CEAB (Canadian
Engineering Accreditation Board) Learning Outcomes (LOs) will be assessed based
on your reflections.}

\section{Changes in Response to Feedback}

\plt{Summarize the changes made over the course of the project in response to
feedback from TAs, the instructor, teammates, other teams, the project
supervisor (if present), and from user testers.}

\plt{For those teams with an external supervisor, please highlight how the feedback 
from the supervisor shaped your project.  In particular, you should highlight the 
supervisor's response to your Rev 0 demonstration to them.}

\plt{Version control can make the summary relatively easy, if you used issues
and meaningful commits.  If you feedback is in an issue, and you responded in
the issue tracker, you can point to the issue as part of explaining your
changes.  If addressing the issue required changes to code or documentation, you
can point to the specific commit that made the changes.  Although the links are
helpful for the details, you should include a label for each item of feedback so
that the reader has an idea of what each item is about without the need to click
on everything to find out.}

\plt{If you were not organized with your commits, traceability between feedback
and commits will not be feasible to capture after the fact.  You will instead
need to spend time writing down a summary of the changes made in response to
each item of feedback.}

\plt{You should address EVERY item of feedback.  A table or itemized list is
recommended.  You should record every item of feedback, along with the source of
that feedback and the change you made in response to that feedback.  The
response can be a change to your documentation, code, or development process.
The response can also be the reason why no changes were made in response to the
feedback.  To make this information manageable, you will record the feedback and
response separately for each deliverable in the sections that follow.}

\plt{If the feedback is general or incomplete, the TA (or instructor) will not
be able to grade your response to feedback.  In that case your grade on this
document, and likely the Revision 1 versions of the other documents will be
low.} 

\subsection{SRS and Hazard Analysis}

\subsection{Design and Design Documentation}

\subsection{VnV Plan and Report}

\section{Challenge Level and Extras}

\subsection{Challenge Level}

\plt{State the challenge level (advanced, general, basic) for your project.  Your challenge level should exactly match what is included in your problem statement.  This should be the challenge level agreed on between you and the course instructor.}

\subsection{Extras}

\plt{Summarize the extras (if any) that were tackled by this project.  Extras
can include usability testing, code walkthroughs, user documentation, formal
proof, GenderMag personas, Design Thinking, etc.  Extras should have already
been approved by the course instructor as included in your problem statement.}

\section{Design Iteration (LO11 (PrototypeIterate))}

\plt{Explain how you arrived at your final design and implementation.  How did
the design evolve from the first version to the final version?} 

\plt{Don't just say what you changed, say why you changed it.  The needs of the
client should be part of the explanation.  For example, if you made changes in
response to usability testing, explain what the testing found and what changes
it led to.}

\section{Design Decisions (LO12)}

\plt{Reflect and justify your design decisions.  How did limitations,
 assumptions, and constraints influence your decisions?  Discuss each of these
 separately.}

\section{Economic Considerations (LO23)}

\plt{Is there a market for your product? What would be involved in marketing your 
product? What is your estimate of the cost to produce a version that you could 
sell?  What would you charge for your product?  How many units would you have to 
sell to make money? If your product isn't something that would be sold, like an 
open source project, how would you go about attracting users?  How many potential 
users currently exist?}

\section{Reflection on Project Management (LO24)}

\subsection{How Does Your Project Management Compare to Your Development Plan}

With regards to the team meeting plan, we were hoping to be able to meet at least 4 hours every week
including the 2 hour tutorial time alloted to us for the capstone course. We met this goal for the most
with the exception of a few weeks where the team had midterms or other commitments. 
We also had a few weeks where we met for more than 4 hours to catch up on work. Over all,
The team was satisfied with how much we met and collaborated with each other.

Th team adhered bu the team communication plan where we used github issues to track our
progress and communicate with each other. We also used discord to communicate with each other.
By assigning the right tags to the issues, we were able to track our progress and communicate with each other effectively.

When it came to team roles, Mohsin as the team leader always ensured the team was on the right
track when it came to the projects needs and ensured everyones opinions were heard. Mohsin also made 
an effort to always take notes during team and TA meetings which the etam could reflect back upon.
He was also the team liason and was always the one communicating with the TA, prof and our supervisor
when it came to any questions or concerns. Among these things, Mohsin also helped develop the product
by contributing the frontend and setup the frontend testing framework.

Lucas's expertise in frontend design came in handy and helped a lot with the UI/UX of the product. He also
ensured that the team was following the git flow plan setup, everything from issue creation to pull requests and merging.
He also lead the development of the POC demo and setup the majority of the cloud infrastructure the team used.

Dennis and Julian worked on the model used to detect plagiarism. Both of their experitse in
this field came in handy wether that be making test cases to etst the model or actually developing and enhancing the model.

Luigi's knowledge in MOSS and plagiarism detection helped the team a lot when it came to understanding how plagiarism detection works.
He also helped with the development of the model by integrating line by line detection.

Along with what was mentioned above, every team memeber contributed to bpth the product's frontend and backend.

The expected technology listed was Git, GitHub, Python, VSCode and LaTeX which the team used along 
with other technologies such as React for the frontend, ExpressJS for the backend, and AWS services 
such as S3 for storage and SQS for queueing. The team also used GitHub Actions for CI/CD and testing.


\subsection{What Went Well?}

The team successfully adhered to the development plan for the most part, ensuring consistent communication and collaboration. The use of GitHub issues for tracking progress and assigning tasks proved to be highly effective, as it allowed the team to stay organized and maintain transparency. Discord was also a valuable tool for quick communication and discussions.

The adoption of Git flow ensured a smooth workflow for version control, with clear processes for issue creation, pull requests, and merging. This minimized conflicts and streamlined the development process.

From a technological perspective, the team effectively utilized the planned tools, including Git, GitHub, Python, VSCode, and LaTeX. The integration of additional technologies, such as React for the frontend, ExpressJS for the backend, and AWS services like S3 and SQS, enhanced the product's functionality and scalability. GitHub Actions for CI/CD and testing further improved the development pipeline by automating builds and tests.

The team's ability to meet regularly, collaborate effectively, and leverage each member's expertise contributed significantly to the project's success. The division of roles and responsibilities was clear, and each member's contributions aligned well with their strengths, ensuring steady progress throughout the project.

\subsection{What Went Wrong?}

\plt{What went wrong in terms of processes and technology?}

\subsection{What Would you Do Differently Next Time?}

\plt{What will you do differently for your next project?}

\section{Reflection on Capstone}

\plt{This question focuses on what you learned during the course of the capstone project.}

\subsection{Which Courses Were Relevant}

\plt{Which of the courses you have taken were relevant for the capstone project?}

\subsection{Knowledge/Skills Outside of Courses}

\plt{What skills/knowledge did you need to acquire for your capstone project
that was outside of the courses you took?}

\end{document}